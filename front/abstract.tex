% !TeX root = ../main.tex

\begin{abstract}

中文摘要中文摘要中文摘要中文摘要中文摘要中文摘要中文摘要中文摘要中文摘要中文摘要中文摘要中文摘要中文摘要中文摘要中文摘要中文摘要中文摘要中文摘要中文摘要中文摘要中文摘要中文摘要中文摘要中文摘要中文摘要中文摘要中文摘要中文摘要中文摘要中文摘要中文摘要中文摘要中文摘要中文摘要中文摘要中文摘要中文摘要中文摘要中文摘要中文摘要中文摘要中文摘要中文摘要中文摘要中文摘要中文摘要中文摘要中文摘要中文摘要中文摘要中文摘要中文摘要中文摘要中文摘要中文摘要中文摘要中文摘要中文摘要中文摘要中文摘要中文摘要中文摘要中文摘要中文摘要中文摘要中文摘要中文摘要中文摘要中文摘要中文摘要中文摘要中文摘要中文摘要中文摘要中文摘要中文摘要中文摘要中文摘要中文摘要

\end{abstract}

\begin{abstract*}

    The implied barrier is the barrier that yields a default probability or 
    reproduces a barrier option price. This thesis proposes a highly efficient 
    algorithm based on the Black-Scholes model to find the implied barrier and proves 
    its convergence rate. In particular, it presents an $O(n\,\text{log}\,n)$-time 
    tree algorithm to find the implied step barrier, where $n$ denotes the tree’s 
    number of time steps. Our algorithm is one order faster than the standard tree 
    algorithm. This algorithm can be easily adapted to obtain the implied 
    barrier when it is defined at discrete points, or to price various types of 
    barrier options with accuracy. For the case of a constant barrier, the 
    implied barrier is proved to converge to its true value at a rate of $O(n^{-1})$. 
    Numerical experiments confirm this rate. The experiments also show that the 
    implied step barrier converges at the same rate. The implied barrier can be seen 
    as the default boundary that reproduces the default probabilities under a 
    structural model. When the default probabilities are obtained from publicly 
    available data, the implied barrier reflects the market consensus. Finally, we 
    propose (1) a new binary default prediction scheme for publicly listed 
    corporations and (2) a new evaluation scheme for binary default prediction 
    schemes. The results show good performance for publicly listed U.S. companies 
    during 1991--2023.

\end{abstract*}