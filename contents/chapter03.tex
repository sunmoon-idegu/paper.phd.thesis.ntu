% !TeX root = ../main.tex

\chapter{Algorithms}\label{chap:algorithms}
    
    This chapter introduces the algorithm for finding the implied barrier and its 
    adaptations. Section~\ref{sec:algorithms-core} describes the core algorithm for 
    finding the implied step barrier and its time complexity.
    Section~\ref{sec:algorithms-adaptations} explains how the core algorithm can be
    applied to barrier-option pricing. It then shows how the core algorithm can be 
    adapted to different types of moving barriers. Finally, we discuss how to adjust 
    the algorithm to deal with time-varying models. 
    Section~\ref{sec:algorithms-experiments} presents the numerical results of the core 
    algorithm and its adaptations. 

\section{The Core Algorithm}\label{sec:algorithms-core}
    
    Our core algorithm works with a step barrier. We begin by introducing the 
    notations and terminology used in this chapter. A step barrier consists of $m$ 
    barrier levels for contiguous yet disjoint time intervals $(\,t_0, t_0+t_1\,]$, 
    $(\,t_0+t_1, t_0+t_2\,]$, $\ldots$, $(\,t_0+t_{m-1}, t_0+t_m\,]$. Each barrier 
    level is left-continuous but not right-continuous by this convention. In this 
    chapter, we let $t_0$ represent the current time, which makes $t_0 = 0$ (see
    Figure~\ref{fig:step_barrier}). The core algorithm works with one BTT per time 
    interval to find the barrier levels $b_1$, $b_2$, $\ldots$, $b_m$ that give rise 
    to the barrier-hitting probabilities $p_1$, $p_2$, \ldots, $p_m$ observed at time 
    $t_0$ (see Figure~\ref{fig:multiple_BTT}). Specifically, $p_i$ denotes the 
    probability of the stock price hitting the barrier on or before the monitored 
    time point $t_0+t_i$. We assume $p_1 < p_2 < \cdots < p_m$ to avoid 
    easy-to-handle degenerate cases and add $p_0 = 0$ for convenience.
    
    \begin{figure}[!t]
        \centering
        \begin{tikzpicture}
            \begin{axis}[
                    width=0.8\textwidth,
                    height=0.6\textwidth,
                    xtick = {0.5,1.5,2.5,3.5,4.5,5.5},
                    xticklabels = {$t_0$,$t_1$,$t_2$,$\ldots$,$t_{m-1}$,$t_m$},
                    xticklabel style={text height=1.5ex, font=\large},
                    ytick = {2.4,0.8,2},
                    yticklabels = {$b_1$,$b_2$,$b_m$},
                    %yticklabels = {$ $,$ $,$ $},
                    yticklabel style={font=\large},
                    xmin=-0.5,xmax=6,
                    ymin=0,ymax=3.6,
                ]
                \addplot[line,-,domain=0.5:1.5]{2.4}; %node [midway,above] {$b_1$};
                \addplot[line,-,domain=1.5:2.5]{0.8}; %node [midway,above] {$b_2$};
                \node at (axis cs:3.5,1.6) {$\ldots \ldots$};
                \addplot[line,-,domain=4.5:5.5]{2};   %node [midway,above] {$b_m$};
                \addplot[hollow]coordinates{(0.5,2.4)(1.5,0.8)(4.5,2)};
                \addplot[solid] coordinates{(1.5,2.4)(2.5,0.8)(5.5,2)};
            \end{axis}
        \end{tikzpicture}
        \caption{A step barrier with $m$ time intervals.}
        \label{fig:step_barrier}
    \end{figure}

    \begin{figure}[!t]
        \centering
        \resizebox{0.9\textwidth}{!}{
            \begin{tikzpicture}[myset/.list={
                (12,1),
                (11,2),(13,2),(15,2),
                (10,3),(12,3),(14,3),(16,3),
                (9,4),(11,4),(13,4),(15,4),(17,4),
                (3,8),(5,8),(7,8),(9,8),(11,8),(13,8),
                (2,9),(4,9),(6,9),(8,9),(10,9),(12,9),(14,9),
                (1,10),(3,10),(5,10),(7,10),(9,10),(11,10),(13,10),(15,10)},
                fillset/.list={(17,4),(13,10)}, %(12,7)
                left extended/.style={shorten <=-#1},
                left extended/.default=1cm,
                right extended/.style={shorten >=-#1},
                right extended/.default=1cm]
            \matrix (tree)[%
                nodes in empty cells,
                matrix of nodes,
                nodes={anchor=center,minimum size=1cm},
                column sep=1.5cm,
                row sep=0.5cm,
                ampersand replacement=\&]
            {
                  \&     \&     \&     \&     \&     \&     \&     \&     \&     \\
                  \&     \&     \&     \&     \&     \&     \&     \&     \&     \\
                  \&     \&     \&     \&     \&     \&     \&     \&     \&     \\
                  \&     \&     \&     \&     \&     \&     \&     \&     \&     \\
                  \&     \&     \&     \&     \&     \&     \&     \&     \&     \\
                  \&     \&     \&     \&     \&     \&     \&     \&     \&     \\
                  \&     \&     \&     \&     \&     \&     \&     \&     \&     \\
                  \&     \&     \&     \&     \&     \&     \&     \&     \&     \\
                  \&     \&     \&     \&     \&     \&     \&     \&     \&     \\
                  \&     \&     \&     \&     \&     \&     \&     \&     \&     \\
                  \&     \&     \&     \&     \&     \&     \&     \&     \&     \\
                  \&     \&     \&     \&     \&     \&     \&     \&     \&     \\
                  \&     \&     \&     \&     \&     \&     \&     \&     \&     \\
                  \&     \&     \&     \&     \&     \&     \&     \&     \&     \\
                  \&     \&     \&     \&     \&     \&     \&     \&     \&     \\
                  \&     \&     \&     \&     \&     \&     \&     \&     \&     \\
                  \&     \&     \&     \&     \&     \&     \&     \&     \&     \\
                  \&     \&     \&     \&     \&     \&     \&     \&     \&     \\
              $|$ \&     \&     \& $|$ \&     \&     \& $|$ \&     \&     \& $|$ \\
            };
            \begin{pgfonlayer}{background}
                \draw[line width=0.5mm, left extended] (tree-17-1) -- (tree-17-4);
                \draw[line width=0.5mm, left extended] (tree-12-5.south) -- (tree-12-7.south);
                \draw[line width=0.5mm, left extended] (tree-13-8) -- (tree-13-10);
                \foreach \y in {2,3,4}{
                    \draw[dashed, left extended, right extended] (tree-9-\y) -- (tree-17-\y);
                  }
                \foreach \x in {9,...,17}{
                    \draw[dashed, left extended, right extended] (tree-\x-2) -- (tree-\x-4);
                  }
                \foreach \y in {5,6,7}{
                    \draw[dashed, left extended, right extended] (tree-4-\y) -- (tree-18-\y);
                  }
                \foreach \x in {4,...,18}{
                    \draw[dashed, left extended, right extended] (tree-\x-5.south) -- (tree-\x-7.south);
                  }
                \foreach \y in {8,9,10}{
                    \draw[dashed, left extended, right extended] (tree-1-\y) -- (tree-15-\y);
                  }
                \foreach \x in {1,...,15}{
                    \draw[dashed, left extended, right extended] (tree-\x-8) -- (tree-\x-10);
                  }
            \end{pgfonlayer}
            \node[draw,circle,minimum size=0.75cm,fill=white,font=\large,anchor=center] (a) at (tree-6-5.south) {};
            \node[draw,circle,minimum size=0.75cm,fill=white,font=\large,anchor=center] (b) at (tree-8-5.south) {};
            \node[draw,circle,minimum size=0.75cm,fill=white,font=\large,anchor=center] (c) at (tree-10-5.south) {};
            \node[draw,circle,minimum size=0.75cm,fill=white,font=\large,anchor=center] (d) at (tree-12-5.south) {};
            \node[draw,circle,minimum size=0.75cm,fill=white,font=\large,anchor=center] (e) at (tree-14-5.south) {};
            \node[draw,circle,minimum size=0.75cm,fill=white,font=\large,anchor=center] (f) at (tree-16-5.south) {};
            \node[draw,circle,minimum size=0.75cm,fill=white,font=\large,anchor=center] (g) at (tree-5-6.south) {};
            \node[draw,circle,minimum size=0.75cm,fill=white,font=\large,anchor=center] (h) at (tree-7-6.south) {};
            \node[draw,circle,minimum size=0.75cm,fill=white,font=\large,anchor=center] (i) at (tree-9-6.south) {};
            \node[draw,circle,minimum size=0.75cm,fill=white,font=\large,anchor=center] (j) at (tree-11-6.south) {};
            \node[draw,circle,minimum size=0.75cm,fill=white,font=\large,anchor=center] (k) at (tree-13-6.south) {};
            \node[draw,circle,minimum size=0.75cm,fill=white,font=\large,anchor=center] (l) at (tree-15-6.south) {};
            \node[draw,circle,minimum size=0.75cm,fill=white,font=\large,anchor=center] (m) at (tree-17-6.south) {};
            \node[draw,circle,minimum size=0.75cm,fill=white,font=\large,anchor=center] (n) at (tree-4-7.south) {};
            \node[draw,circle,minimum size=0.75cm,fill=white,font=\large,anchor=center] (o) at (tree-6-7.south) {};
            \node[draw,circle,minimum size=0.75cm,fill=white,font=\large,anchor=center] (p) at (tree-8-7.south) {};
            \node[draw,circle,minimum size=0.75cm,fill=white,font=\large,anchor=center] (q) at (tree-10-7.south) {};
            \node[draw,circle,minimum size=0.75cm,fill=black,font=\large,anchor=center] (r) at (tree-12-7.south) {};
            \node[draw,circle,minimum size=0.75cm,fill=white,font=\large,anchor=center] (s) at (tree-14-7.south) {};
            \node[draw,circle,minimum size=0.75cm,fill=white,font=\large,anchor=center] (t) at (tree-16-7.south) {};
            \node[draw,circle,minimum size=0.75cm,fill=white,font=\large,anchor=center] (u) at (tree-18-7.south) {};
            \node[below=0.5cm,font=\Huge] at (tree-19-1) {$t_0$};
            \node[below=0.5cm,font=\Huge] at (tree-19-4) {$t_1$};
            \node[below=0.5cm,font=\Huge] at (tree-19-7) {$t_2$};
            \node[below=0.5cm,font=\Huge] at (tree-19-10) {$t_3$};
            \draw[-{Latex[length=2mm]}] (tree-12-1) -- (tree-11-2); 
            \draw[-{Latex[length=2mm]}] (tree-12-1) -- (tree-13-2); 
            \draw[-{Latex[length=2mm]}] (tree-12-1) -- (tree-15-2); 
            \draw[-{Latex[length=2mm]}] (tree-11-2) -- (tree-10-3); 
            \draw[-{Latex[length=2mm]}] (tree-11-2) -- (tree-12-3); 
            \draw[-{Latex[length=2mm]}] (tree-13-2) -- (tree-12-3); 
            \draw[-{Latex[length=2mm]}] (tree-13-2) -- (tree-14-3); 
            \draw[-{Latex[length=2mm]}] (tree-15-2) -- (tree-14-3); 
            \draw[-{Latex[length=2mm]}] (tree-15-2) -- (tree-16-3); 
            \draw[-{Latex[length=2mm]}] (tree-10-3) -- (tree-9-4); 
            \draw[-{Latex[length=2mm]}] (tree-10-3) -- (tree-11-4); 
            \draw[-{Latex[length=2mm]}] (tree-12-3) -- (tree-11-4); 
            \draw[-{Latex[length=2mm]}] (tree-12-3) -- (tree-13-4); 
            \draw[-{Latex[length=2mm]}] (tree-14-3) -- (tree-13-4); 
            \draw[-{Latex[length=2mm]}] (tree-14-3) -- (tree-15-4); 
            \draw[-{Latex[length=2mm]}] (tree-16-3) -- (tree-15-4); 
            \draw[-{Latex[length=2mm]}] (tree-16-3) -- (tree-17-4); 
            \draw[-{Latex[length=2mm]}] (tree-9-4) --  (a);  
            \draw[-{Latex[length=2mm]}] (tree-9-4) --  (b);
            \draw[-{Latex[length=2mm]}] (tree-9-4) --  (c); 
            \draw[-{Latex[length=2mm]}] (tree-11-4) -- (b);
            \draw[-{Latex[length=2mm]}] (tree-11-4) -- (c); 
            \draw[-{Latex[length=2mm]}] (tree-11-4) -- (d); 
            \draw[-{Latex[length=2mm]}] (tree-13-4) -- (c); 
            \draw[-{Latex[length=2mm]}] (tree-13-4) -- (d); 
            \draw[-{Latex[length=2mm]}] (tree-13-4) -- (e); 
            \draw[-{Latex[length=2mm]}] (tree-15-4) -- (d); 
            \draw[-{Latex[length=2mm]}] (tree-15-4) -- (e); 
            \draw[-{Latex[length=2mm]}] (tree-15-4) -- (f); 
            \draw[-{Latex[length=2mm]}] (a) -- (g); 
            \draw[-{Latex[length=2mm]}] (a) -- (h); 
            \draw[-{Latex[length=2mm]}] (b) -- (h); 
            \draw[-{Latex[length=2mm]}] (b) -- (i); 
            \draw[-{Latex[length=2mm]}] (c) -- (i); 
            \draw[-{Latex[length=2mm]}] (c) -- (j); 
            \draw[-{Latex[length=2mm]}] (d) -- (j); 
            \draw[-{Latex[length=2mm]}] (d) -- (k); 
            \draw[-{Latex[length=2mm]}] (e) -- (k); 
            \draw[-{Latex[length=2mm]}] (e) -- (l); 
            \draw[-{Latex[length=2mm]}] (f) -- (l); 
            \draw[-{Latex[length=2mm]}] (f) -- (m); 
            \draw[-{Latex[length=2mm]}] (g) -- (n); 
            \draw[-{Latex[length=2mm]}] (g) -- (o); 
            \draw[-{Latex[length=2mm]}] (h) -- (o); 
            \draw[-{Latex[length=2mm]}] (h) -- (p); 
            \draw[-{Latex[length=2mm]}] (i) -- (p); 
            \draw[-{Latex[length=2mm]}] (i) -- (q); 
            \draw[-{Latex[length=2mm]}] (j) -- (q); 
            \draw[-{Latex[length=2mm]}] (j) -- (r); 
            \draw[-{Latex[length=2mm]}] (k) -- (r); 
            \draw[-{Latex[length=2mm]}] (k) -- (s); 
            \draw[-{Latex[length=2mm]}] (l) -- (s); 
            \draw[-{Latex[length=2mm]}] (l) -- (t); 
            \draw[-{Latex[length=2mm]}] (m) -- (t); 
            \draw[-{Latex[length=2mm]}] (m) -- (u); 
            \draw[-{Latex[length=2mm]}] (n) -- (tree-3-8); 
            \draw[-{Latex[length=2mm]}] (n) -- (tree-5-8); 
            \draw[-{Latex[length=2mm]}] (n) -- (tree-7-8); 
            \draw[-{Latex[length=2mm]}] (o) -- (tree-5-8); 
            \draw[-{Latex[length=2mm]}] (o) -- (tree-7-8); 
            \draw[-{Latex[length=2mm]}] (o) -- (tree-9-8); 
            \draw[-{Latex[length=2mm]}] (p) -- (tree-7-8); 
            \draw[-{Latex[length=2mm]}] (p) -- (tree-9-8); 
            \draw[-{Latex[length=2mm]}] (p) -- (tree-11-8); 
            \draw[-{Latex[length=2mm]}] (q) -- (tree-9-8); 
            \draw[-{Latex[length=2mm]}] (q) -- (tree-11-8); 
            \draw[-{Latex[length=2mm]}] (q) -- (tree-13-8); 
            \draw[-{Latex[length=2mm]}] (tree-3-8) -- (tree-2-9); 
            \draw[-{Latex[length=2mm]}] (tree-3-8) -- (tree-4-9); 
            \draw[-{Latex[length=2mm]}] (tree-5-8) -- (tree-4-9); 
            \draw[-{Latex[length=2mm]}] (tree-5-8) -- (tree-6-9); 
            \draw[-{Latex[length=2mm]}] (tree-7-8) -- (tree-6-9); 
            \draw[-{Latex[length=2mm]}] (tree-7-8) -- (tree-8-9); 
            \draw[-{Latex[length=2mm]}] (tree-9-8) -- (tree-8-9); 
            \draw[-{Latex[length=2mm]}] (tree-9-8) -- (tree-10-9); 
            \draw[-{Latex[length=2mm]}] (tree-11-8) -- (tree-10-9); 
            \draw[-{Latex[length=2mm]}] (tree-11-8) -- (tree-12-9); 
            \draw[-{Latex[length=2mm]}] (tree-13-8) -- (tree-12-9); 
            \draw[-{Latex[length=2mm]}] (tree-13-8) -- (tree-14-9); 
            \draw[-{Latex[length=2mm]}] (tree-2-9) -- (tree-1-10); 
            \draw[-{Latex[length=2mm]}] (tree-2-9) -- (tree-3-10); 
            \draw[-{Latex[length=2mm]}] (tree-4-9) -- (tree-3-10); 
            \draw[-{Latex[length=2mm]}] (tree-4-9) -- (tree-5-10); 
            \draw[-{Latex[length=2mm]}] (tree-6-9) -- (tree-5-10); 
            \draw[-{Latex[length=2mm]}] (tree-6-9) -- (tree-7-10); 
            \draw[-{Latex[length=2mm]}] (tree-8-9) -- (tree-7-10); 
            \draw[-{Latex[length=2mm]}] (tree-8-9) -- (tree-9-10); 
            \draw[-{Latex[length=2mm]}] (tree-10-9) -- (tree-9-10); 
            \draw[-{Latex[length=2mm]}] (tree-10-9) -- (tree-11-10); 
            \draw[-{Latex[length=2mm]}] (tree-12-9) -- (tree-11-10); 
            \draw[-{Latex[length=2mm]}] (tree-12-9) -- (tree-13-10); 
            \draw[-{Latex[length=2mm]}] (tree-14-9) -- (tree-13-10); 
            \draw[-{Latex[length=2mm]}] (tree-14-9) -- (tree-15-10); 
            \draw (tree-19-1.mid) -- (tree-19-10.mid);
        \end{tikzpicture}}
        \caption{A tree with three BTTs. The BTTs cover time intervals 
                 $(\, t_0, t_1 \,]$, $(\, t_1, t_2 \,]$, and $(\, t_2, t_3 \,]\,$.} 
        \label{fig:multiple_BTT}
    \end{figure}
    
    Assume $\Delta t$ divides $t_i - t_{i-1}$ for all $1 \leq i \leq m$. This can be
    enforced by rounding each $t_i$ to the nearest multiple of $\Delta t$. Let
    $n_i \equiv (t_i - t_{i-1})/\Delta t -1$ denote the number of time steps from
    time $t_{i-1}+\Delta t$ through time $t_i$. So there are $n_i+1$ time steps in
    time interval $(\, t_{i-1}, t_i \,]$, and $t_m-t_0$ is divided into
    $n = \sum_{i=1}^m (n_i+1)$ time steps. Note that $n_i = O(n/m)$. For brevity,
    assume all $n_i$s are even.

    We adopt the following terms. The \emph{survival probability at time $t_i$}, 
    $1-p_i$, is the probability of the stock price not hitting the barrier between 
    times $t_0$ and $t_i$. By the \emph{survival probability of a node} we mean the 
    probability of that node being reached from the root node at time $t_0$ without 
    ever hitting the barrier in the process. A node's 
    \emph{time-$t$ survival probability} denotes that node's survival probability at 
    some \emph{future} time $t$. 
    
    The core algorithm iterates the following two parts for $1 \leq i \leq m\,$: (1) 
    Find the barrier level $b_i$ for the time interval $(\, t_{i-1}, t_i \,]$ that 
    yields $1-p_i$, the survival probability at time $t_i$, and (2) propagate the 
    survival probabilities of the nodes above the barrier level $b_{i-1}$ at time 
    $t_{i-1}$ to the nodes above the barrier level $b_i$ at time $t_i$. The next two 
    Subsections, \ref{sec:algorithms-core-part1} and \ref{sec:algorithms-core-part2}, 
    address these two parts, respectively.

\subsection{Finding the Exact Implied Barrier Level}\label{sec:algorithms-core-part1}

    We will describe part (1) of the core algorithm with reference to 
    Figure~\ref{fig:algor_part_1}. The labels $\text{R}_0$--$\text{R}_2$ and 
    $\text{A}_0$--$\text{A}_4$ there refer to both the nodes and their associated 
    probabilities (to be specified later) for economy of notations. By induction, 
    the survival probabilities of the nodes at time $t_{i-1}$ (nodes 
    $\text{R}_0$--$\text{R}_2$) have been obtained in the previous iteration. Now 
    grow a BTT from those nodes through time $t_i$. For efficiency's sake, this tree 
    must not actually be built. (In contrast, what we shall call standard algorithms 
    build trees and apply forward induction to propagate the survival probabilities 
    node by node; they are otherwise identical to the core algorithm.) Place the 
    middle branch of each trinomial structure (nodes $\text{A}_1$--$\text{A}_3$) so 
    they all have the same logarithmic return of $\theta-\sigma\sqrt{\Delta t}\,$. 
    That means $\beta = -\sigma\sqrt{\Delta t} \,$, $P_u = P_m = 0.5$, and $P_d = 0$ 
    by formulas~(\ref{eq:pu})--(\ref{eq:pd}). Line up the lattice for the binomial 
    structure with those nodes at time $t_{i-1}+\Delta t$ (nodes 
    $\text{A}_0$--$\text{A}_4$). This positioning of the underlying lattice, thus 
    the binomial structure as well, will be fine-tuned later by adjusting $\beta$. 
    
    \begin{figure}[!t]
        \centering
        \resizebox{0.9\textwidth}{!}{
            \begin{tikzpicture}[myset/.list={%(6,1),(8,1),(10,1),
                (3,2),(5,2),(7,2),(9,2),(11,2),
                (2,3),(4,3),(6,3),(8,3),(10,3),(12,3),
                (1,4),(3,4),(5,4),(7,4),(9,4),(11,4),(13,4)},
                  left extended/.style={shorten <=-#1},
                  left extended/.default=1cm,
                  right extended/.style={shorten >=-#1},
                  right extended/.default=1cm]
          \matrix (tree) [%
            nodes in empty cells,
            matrix of nodes,
            nodes={anchor=center,minimum size=0.75cm},
            column sep=2cm,
            row sep=0.25cm,
            ampersand replacement=\&
          ]
          {
                            \&              \& \& $\text{N}_6$  \\
                            \&              \& \&   \\
                            \& $\text{A}_4$ \& \& $\text{N}_5$  \\
                            \&              \& \&   \\
                            \& $\text{A}_3$ \& \& $\text{N}_4$  \\
                            \&              \& \&   \\
                            \& $\text{A}_2$ \& \& $\text{N}_3$  \\
                            \&              \& \&   \\
                            \& $\text{A}_1$ \& \& $\text{N}_2$  \\
                            \&              \& \&   \\
                            \& $\text{A}_0$ \& \& $\text{N}_1$  \\
                            \&              \& \&   \\
                            \&              \& \& $\text{N}_0$  \\
                        $|$ \& $|$ \& $|$ \& $|$ \\
          };
          \begin{pgfonlayer}{background}
            \foreach \y in {2,3}{
                \draw[dashed, left extended, right extended] (tree-1-\y) -- (tree-13-\y);
              }
            \draw[dashed, left extended=1.23cm, right extended=1.25cm] (tree-1-4) -- (tree-13-4);
            \foreach \x in {1,2,4,6,8,10,12,13}{
                \draw[dashed, left extended, right extended=1.07cm] (tree-\x-2) -- (tree-\x-4);
              }
            \draw[dashed, left extended, right extended=1.243cm] (tree-1-2) -- (tree-1-4);
            \draw[dashed, left extended, right extended=1.243cm] (tree-13-2) -- (tree-13-4);
            \foreach \x in {3,5,7,9,11}{
                \draw[dashed, left extended=1.18cm, right extended=1.235cm] (tree-\x-2) -- (tree-\x-4);
              }
          \end{pgfonlayer}
          \node[below=0.5cm,font=\large] at (tree-14-1) {$t_{i-1}$};
          \node[below=0.5cm,font=\large] at (tree-14-2) {$t_{i-1}+\Delta t$};
          \node[below=0.5cm,font=\large] at (tree-14-3) {$t_{i-1}+2\Delta t$};
          \node[below=0.5cm,font=\large] at (tree-14-4) {$t_i$};
          \node[left=1cm, font=\large] at (tree-5-1.north) {$\ldots \ldots$};
          \node[left=1cm, font=\large] at (tree-7-1.north) {$\ldots \ldots$};
          \node[left=1cm, font=\large] at (tree-9-1.north) {$\ldots \ldots$};
          \node[right=1cm,font=\large] at (tree-3-4.north) {$\ldots \ldots$};
          \node[right=1cm,font=\large] at (tree-7-4.north) {$\ldots \ldots$};
          \node[right=1cm,font=\large] at (tree-11-4.north) {$\ldots \ldots$};
          \node[draw,circle,minimum size=0.75cm,fill=white,font=\large,anchor=center] (a) at (tree-5-1.north) {$\text{R}_2$};
          \node[draw,circle,minimum size=0.75cm,fill=white,font=\large,anchor=center] (b) at (tree-7-1.north) {$\text{R}_1$};
          \node[draw,circle,minimum size=0.75cm,fill=white,font=\large,anchor=center] (c) at (tree-9-1.north) {$\text{R}_0$};
          \draw[-{Latex[length=2mm]}] (a)  -- (tree-3-2)         node [pos=.1,above=1mm, font=\large] {$P_u$}; 
          \draw[-{Latex[length=2mm]}] (a)  -- (tree-5-2)         node [pos=.25,above,font=\large] {$P_m$}; 
          \draw[dashed,-{Latex[length=2mm]}] (a)  -- (tree-7-2)  node [pos=.1,below, font=\large] {$P_d$}; 
          \draw[-{Latex[length=2mm]}] (b)  -- (tree-5-2)         node [pos=.1,above=1mm, font=\large] {$P_u$}; 
          \draw[-{Latex[length=2mm]}] (b)  -- (tree-7-2)         node [pos=.25,above,font=\large] {$P_m$}; 
          \draw[dashed,-{Latex[length=2mm]}] (b)  -- (tree-9-2)  node [pos=.1,below, font=\large] {$P_d$}; 
          \draw[-{Latex[length=2mm]}] (c) -- (tree-7-2)          node [pos=.1,above=1mm, font=\large] {$P_u$}; 
          \draw[-{Latex[length=2mm]}] (c) -- (tree-9-2)          node [pos=.25,above,font=\large] {$P_m$}; 
          \draw[dashed,-{Latex[length=2mm]}] (c) -- (tree-11-2)  node [pos=.1,below, font=\large] {$P_d$}; 
          \draw[-{Latex[length=2mm]}] (tree-3-2) -- (tree-2-3); 
          \draw[-{Latex[length=2mm]}] (tree-3-2) -- (tree-4-3); 
          \draw[-{Latex[length=2mm]}] (tree-5-2) -- (tree-4-3); 
          \draw[-{Latex[length=2mm]}] (tree-5-2) -- (tree-6-3); 
          \draw[-{Latex[length=2mm]}] (tree-7-2) -- (tree-6-3); 
          \draw[-{Latex[length=2mm]}] (tree-7-2) -- (tree-8-3); 
          \draw[-{Latex[length=2mm]}] (tree-9-2) -- (tree-8-3); 
          \draw[-{Latex[length=2mm]}] (tree-9-2) -- (tree-10-3); 
          \draw[-{Latex[length=2mm]}] (tree-11-2) -- (tree-10-3); 
          \draw[-{Latex[length=2mm]}] (tree-11-2) -- (tree-12-3); 
          \draw[-{Latex[length=2mm]}] (tree-2-3) -- (tree-1-4); 
          \draw[-{Latex[length=2mm]}] (tree-2-3) -- (tree-3-4); 
          \draw[-{Latex[length=2mm]}] (tree-4-3) -- (tree-3-4); 
          \draw[-{Latex[length=2mm]}] (tree-4-3) -- (tree-5-4); 
          \draw[-{Latex[length=2mm]}] (tree-6-3) -- (tree-5-4); 
          \draw[-{Latex[length=2mm]}] (tree-6-3) -- (tree-7-4); 
          \draw[-{Latex[length=2mm]}] (tree-8-3) -- (tree-7-4); 
          \draw[-{Latex[length=2mm]}] (tree-8-3) -- (tree-9-4); 
          \draw[-{Latex[length=2mm]}] (tree-10-3) -- (tree-9-4); 
          \draw[-{Latex[length=2mm]}] (tree-10-3) -- (tree-11-4); 
          \draw[-{Latex[length=2mm]}] (tree-12-3) -- (tree-11-4); 
          \draw[-{Latex[length=2mm]}] (tree-12-3) -- (tree-13-4); 
          \draw[-{Latex[length=2mm]}] (tree-14-1.mid) -- (tree-14-2.mid) ;
          \draw[-{Latex[length=2mm]}] (tree-14-2.mid) -- (tree-14-3.mid) ;
          \draw[-{Latex[length=2mm]}] (tree-14-3.mid) -- (tree-14-4.mid) ;
      \end{tikzpicture}}
        \caption{The initial BTT with $n_i = 2$. The dashed lines have zero 
                 probabilities $P_d$.}
        \label{fig:algor_part_1}
    \end{figure}
   
    Next, perform a binary search on the tree nodes at time $t_i$ (nodes 
    $\text{N}_0$--$\text{N}_6$) as the barrier levels. It starts with the interval 
    spanning the lowest barrier (node $\text{N}_0$) through the upmost barrier 
    (node $\text{N}_6$). The lowest barrier has no possibility of being hit with 
    $P_d = 0$, while survival is impossible with the upmost barrier. They thus yield 
    survival probabilities $0$ and $1-p_{i-1}$ at time $t_i$, respectively, which 
    automatically bracket the target $1-p_i$ as $p_{i-1} < p_i$. The search then 
    picks the middle node at time $t_i$ as barrier and calculates the resulting 
    survival probability at time $t_i$ (to be described in the following paragraphs). 
    It continues with the subinterval whose survival probabilities at time $t_i$ 
    bracket $1-p_i$ and repeats the halving procedure. Stop when the subinterval 
    narrows down to two \emph{adjacent} barrier levels, say $\text{N}_2$ and 
    $\text{N}_3$. 

    We return to obtaining the survival probability at time $t_i$ induced by a 
    selected barrier level. Consider any node above the barrier at time $t_{i-1}$. 
    Sum the time-$t_i$ survival probabilities of its three successor nodes at time 
    $t_{i-1}+\Delta t$ weighted by the respective branching probabilities $P_u$, 
    $P_m$, and $P_d$, and then multiply this sum by the node's survival probability.
    Finally, add up such sums to obtain the desired survival probability at time 
    $t_i$. For example, suppose $\text{N}_0$ is the barrier level. The survival 
    probability at time $t_i$ equals
    \begin{align*}
               &\left(\text{A}_0P_d+\text{A}_1P_m+\text{A}_2P_u\right)\text{R}_0 \\
        +\quad &\left(\text{A}_1P_d+\text{A}_2P_m+\text{A}_3P_u\right)\text{R}_1 \\
        +\quad &\left(\text{A}_2P_d+\text{A}_3P_m+\text{A}_4P_u\right)\text{R}_2, \numberthis \label{eq:survival_prob_t_case0}
    \end{align*}
    where $\text{R}_0$, $\text{R}_1$, $\text{R}_2$ double as these nodes' survival 
    probabilities, and $\text{A}_0$--$\text{A}_4$ double as these nodes' time-$t_i$ 
    survival probabilities. We next show how to efficiently calculate (i) the 
    time-$t_i$ survival probabilities (like $\text{A}_0$--$\text{A}_4$) and (ii) the 
    survival probability at time $t_i$ (like 
    formula~(\ref{eq:survival_prob_t_case0})). 

    We start with task (i). Consider any node above the barrier at time 
    $t_{i-1}+\Delta t$. Suppose this node needs to take $2 \leq \ell \leq n_i$ 
    consecutive down moves to reach the barrier. We call the number of such moves the 
    \emph{distance to barrier}. Note that $\ell$ is even because $n_i$ is even. In 
    Figure~\ref{fig:bino_tri_lattice_with_barrier}, for example, node B needs 
    $\ell = 4$ consecutive down moves to reach the barrier $L$. Consider paths 
    starting from that node with $n_i-j$ up moves and $j$ down moves that hit the 
    barrier. Their count is $\binom{n_i}{j-\ell}$ by substituting $u = \ell$ and 
    $v = n_i-2j+\ell$ into formula~(\ref{eq:reflection_principle}). The 
    time-$t_i$ survival probability of the node is 
    \begin{equation}\label{eq:survival_probability}
        B_\ell\quad \equiv\quad \sum_{j=0}^{\frac{n_i+\ell}{2}-1}\binom{n_i}{j}p^{n_i-j}(1-p)^j-\sum_{j=\ell}^{\frac{n_i+\ell}{2}-1}\binom{n_i}{j-\ell}p^{n_i-j}(1-p)^{j}.
    \end{equation}
    Set $\text{B}_0=0$. Let $G_1$ and $G_2$ be the first and second terms in 
    formula~(\ref{eq:survival_probability}), respectively. The same formula says the 
    time-$t_i$ survival probability of the node immediately below the above-mentioned 
    node, whose distance to barrier is $\ell-2 \geq 0$,
    \begin{align}
        &\scalebox{0.9}{$
            B_{\ell-2}\quad \equiv\quad \sum_{j=0}^{\frac{n_i+\ell-2}{2}-1}\binom{n_i}{j}p^{n_i-j}(1-p)^j - \sum_{j=\ell-2}^{\frac{n_i+\ell-2}{2}-1}\binom{n_i}{j-(\ell-2)}p^{n_i-j}(1-p)^{j}$} \notag \\[5mm]
        &\scalebox{0.9}{$
        =\quad \left[G_1 - \binom{n_i}{\frac{n_i+\ell}{2}-1}p^{\frac{n_i-\ell}{2}+1}(1-p)^{\frac{n_i+\ell}{2}-1}\right] - \left(\frac{p}{1-p}\right)^2\left[G_2 + \binom{n_i}{\frac{n_i-\ell}{2}}p^{\frac{n_i-\ell}{2}}(1-p)^{\frac{n_i+\ell}{2}}\right] $}. 
       \label{eq:survival_probability_rec}
    \end{align}
    
    Among the nodes that are positioned above the barrier at time $t_{i-1}+\Delta t$, 
    not more than $n_i/2$ have time-$t_i$ survival probabilities less than $1$ 
    because at most that many of them have a distance to barrier not exceeding $n_i$. 
    The rest, if there are any, need no calculations because their time-$t_i$ 
    survival probabilities equal $1$. Reserve $B_\ell = 1$ for them. Start with the 
    node whose distance to barrier is less than, but closest to, $n_i$. This node's 
    time-$t_i$ survival probability takes $O(n_i)$ time to compute by 
    formula~(\ref{eq:survival_probability}). Next, apply 
    formula~(\ref{eq:survival_probability_rec}) iteratively to obtain each of the 
    above-mentioned $O(n_i)$ nodes' time-$t_i$ survival probabilities. Since
    formula~(\ref{eq:survival_probability_rec}) takes $O(1)$ time to calculate with 
    $G_1$ and $G_2$ available, it takes $O(n_i)$ time to obtain the time-$t_i$ 
    survival probabilities of all the nodes above the barrier at time 
    $t_{i-1}+\Delta t$. 

    We now describe how task (ii), i.e., the survival probability at time $t_i$ like 
    formula~(\ref{eq:survival_prob_t_case0}), can be efficiently calculated from 
    the $\text{B}_\ell$s with translation invariance. We illustrate the idea with 
    examples. First, restructure formula~(\ref{eq:survival_prob_t_case0}) as
    \begin{align*}
               &\left(\text{R}_0P_d\right)\text{A}_0 \\
        +\quad &\left(\text{R}_0P_m+\text{R}_1P_d\right)\text{A}_1 \\
        +\quad &\left(\text{R}_0P_u+\text{R}_1P_m+\text{R}_2P_d\right)\text{A}_2 \\
        +\quad &\left(\text{R}_1P_u+\text{R}_2P_m\right)\text{A}_3 \\
        +\quad &\left(\text{R}_2P_u\right)\text{A}_4. \numberthis \label{eq:survival_prob_t_case1}
    \end{align*}
    Above, $\text{A}_0$, $\text{A}_1$, $\text{A}_2$, $\text{A}_3$, and $\text{A}_4$ 
    equal $\text{B}_2$, $\text{B}_4$, $\text{B}_6$, $\text{B}_8$, and 
    $\text{B}_{10}$, respectively. The probabilities $\text{A}_0$--$\text{A}_4$ 
    change as the barrier level changes. Fortunately, recalculation is unnecessary 
    because nodes with the same distance to barrier have the same time-$t_i$ survival 
    probability by translation invariance. For example, the time-$t_i$ survival 
    probability of node $\text{A}_2$ with $\text{N}_2$ as the new barrier level 
    equals that of node $\text{A}_0$ with $\text{N}_0$ as the barrier level. So if we 
    raise the barrier from $\text{N}_0$ to $\text{N}_2$, the survival probability at 
    time $t_i$ becomes
    \begin{align*}
               &\left(\text{R}_0P_u+\text{R}_1P_m+\text{R}_2P_d\right)\text{A}_2 \\
        +\quad &\left(\text{R}_1P_u+\text{R}_2P_m\right)\text{A}_3 \\
        +\quad &\left(\text{R}_2P_u\right)\text{A}_4. \numberthis \label{eq:survival_prob_t_case2}
    \end{align*}
    But now each node $\text{A}_k$ being relative to the new level $\text{N}_2$, its 
    time-$t_i$ survival probability will take on that of $\text{A}_{k-2}$ when 
    $\text{N}_0$ was the barrier. So the values of $\text{A}_2$, $\text{A}_3$, and 
    $\text{A}_4$ now equal $\text{B}_2$, $\text{B}_4$, and $\text{B}_6$, 
    respectively, instead of the earlier $\text{B}_6$, $\text{B}_8$, and 
    $\text{B}_{10}$. It takes $O(\, \sum_{j=1}^{i-1}n_j \,)$ steps to calculate the 
    survival probability at time $t_i$ because 
    formulas~(\ref{eq:survival_prob_t_case1}) and (\ref{eq:survival_prob_t_case2}) 
    have as many terms.

    We summarize the running time so far. There are at most 
    $1+\sum_{j=1}^i(n_j+2) = O(n)$ nodes above the barrier at time $t_i$. For each 
    node proposed as barrier level, it takes $O(\, \sum_{j=1}^{i-1}n_j \,)$ steps to 
    calculate the survival probability at time $t_i$. With the bisection method, up 
    to $O(\text{log}\ n)$ barrier levels are considered before the two bracketing 
    nodes are identified. Overall, the algorithm spends 
    $O(\,\sum_{j=1}^{i-1}n_j\ \text{log}\ n\,)$ time to find the two bracketing 
    nodes. 

    There are two ways the binary search may end. First, it stops if the survival
    probability induced by an endpoint of the subinterval as the barrier level 
    matches $1-p_i$. In this case, the algorithm returns that barrier level. 
    Otherwise, the subinterval is delimited strictly by two \emph{adjacent} terminal 
    nodes, say $\text{N}_1$ and $\text{N}_2$. This subinterval brackets the desired 
    implied barrier level within a small interval of width $2\sigma\sqrt{\Delta t}$.
    Still, each remains at best an approximation to $b_i$. 

    Surprisingly, the BTT can find the exact $b_i$ that yields the target probability 
    $1-p_i$ by nudging the underlying lattice with the right amount: It is a simple 
    act of solving a quadratic equation. To begin with, we know $b_i$ must lie 
    between the bracketing nodes (nodes $\text{N}_2$ and $\text{N}_3$). This is 
    because the survival probability at time $t_i$ is a continuous function of 
    $\beta$ for $-\sigma \sqrt{\Delta t} \leq \beta \leq \sigma \sqrt{\Delta t}$ by 
    formulas~(\ref{eq:pu})--(\ref{eq:pd}). It remains to position the barrier level 
    by finding the right $\beta$ within the interval to yield a survival probability 
    at time $t_i$ equal to $1-p_i$. Observe that all the trinomial structures share 
    the same $\beta$ and branching probabilities because they straddle across two 
    lattices with the \emph{same} interlinear spacing $\sigma\sqrt{\Delta t}\,$. By 
    our initial choice of $\beta = -\sigma \sqrt{\Delta t}$, nodes $\text{A}_1$, 
    $\text{A}_2$, and $\text{A}_3$ have logarithmic returns of 
    $\theta - 3\sigma\sqrt{\Delta t}$, $\theta - \sigma\sqrt{\Delta t}$, and 
    $\theta + \sigma\sqrt{\Delta t}$ on node $\text{R}_1$, respectively (the branch 
    to node $\text{A}_1$ has zero probability, however). The same holds for the 
    logarithmic returns of nodes $\text{A}_2$, $\text{A}_3$, and $\text{A}_4$ on node 
    $\text{R}_2$, and so on. This implies that the survival probability at time $t_i$ 
    is a linear function of $P_u$, $P_m$, and $P_d$, thus a quadratic function of 
    $\beta$ by formulas~(\ref{eq:pu})--(\ref{eq:pd}). It takes $O(n_i)$ time to 
    assemble the three coefficients of the quadratic function. Now solve the 
    $\beta \in [\,-\sigma\sqrt{\Delta t}, \sigma\sqrt{\Delta t}\,]$ that equates the 
    quadratic function with $1-p_i$. Because the function is concave and brackets 
    $1-p_i$ at the two endpoints of the subinterval, it has a unique solution 
    $\beta^*$ within range~(\ref{eq:beta_range}) as illustrated in 
    Figure~\ref{fig:lemma1}. (Recall that we started with 
    $\beta = -\sigma\sqrt{\Delta t}$.) The underlying grid will move up by 
    $\beta^* + \sigma\sqrt{\Delta t}$. Finally, the algorithm returns the level of 
    the \emph{lower} end of the final subinterval (node $\text{N}_1$) as the implied 
    barrier level.

    \begin{figure}[!t]
        \centering
        \includegraphics[width=0.47\textwidth]{./figures/binary1.pdf}
        \includegraphics[width=0.47\textwidth]{./figures/binary2.pdf}
        \caption{The default probability as a concave quadratic function of $\beta$.
                 %The function value starts at $p_i$ and ends at $p_{i+1}$ by Lemma~\ref{lemma:identical_BTT}.
                 The function's maximum may occur before (the left panel) or after
                 (the right panel) $\sigma\sqrt{\Delta t}$.}
        \label{fig:lemma1}
    \end{figure}

    In summary, iteration $i$ of part (1) takes time
    \begin{equation*}
        O\left[n_i + \left(\sum_{j=1}^{i-1}n_j\ \text{log}\ n\right) + n_i\right] = O\left(n_i + \sum_{j=1}^{i-1}n_j\ \text{log}\ n\right).
    \end{equation*}
    With $m$ iterations in total, part (1) takes time 
    \begin{equation*}
        O\left[\sum_{i=1}^{m}\left(n_i + \sum_{j=1}^{i-1}n_j\ \text{log}\ n\right)\right] %\\[15pt]
        = O\left[\sum_{i=1}^m \left(\frac{n}{m} + \sum_{j=1}^{i-1}\frac{n\ \text{log}\ n}{m}\right)\right] %\\[15pt]
        = O\left(mn\ \text{log}\ n\right). 
    \end{equation*}
    This time bound can be reduced to $O\left[\,mn\ \text{log}\ (n/m)\,\right]$ by
    recalling that $\text{B}_\ell = 1$ for $\ell > n_i\,$, which simplifies the 
    overwhelming majority of the terms in formula~(\ref{eq:survival_prob_t_case1}). 
    But as $m$ is a constant, this improvement is not pursued further.

\subsection{Propagating the Survival Probabilities}\label{sec:algorithms-core-part2}

    Part (2) obtains the survival probability of every node above $b_i$ at time $t_i$ 
    from those above $b_{i-1}$ at time $t_{i-1}$. We shall accomplish this task 
    efficiently with the combinatorics and FFT. 

    The first step obtains the survival probabilities of the nodes above $b_i$ at 
    $t_{i-1}+\Delta t$. This is easily done by summing the contributions from the 
    survival probabilities of its predecessor nodes at time $t_{i-1}$ weighted by 
    $P_u$, $P_m$, and $P_d$. For example, the survival probability of node 
    $\text{A}_2$ in Figure~\ref{fig:algor_part_1} is
    \begin{equation*}
        \text{R}_0 P_u+\text{R}_1 P_m+\text{R}_2 P_d.
    \end{equation*}
    Let $k_i$ denote the number of nodes \emph{above} $b_i$ at time 
    $t_{i-1}+\Delta t$. Name the survival probabilities of these nodes from the top 
    down $q_{k_i-1}, q_{k_i-2}, \ldots, q_0$. Ignore the barrier $b_i$ for the time 
    being. Since the binomial structure has $n_i$ steps, each node at time 
    $t_{i-1}+\Delta t$ will reach those at time $t_i$ with the binomial distribution 
    $B(n_i, p)$. In totality, those $k_i$ nodes will reach $k_i+n_i$ nodes at time 
    $t_i$. Call the survival probabilities of these nodes $N_{k_i + n_i - 1}$, 
    $N_{k_i + n_i - 2}$, $\ldots$, $N_{0}$, again from the top down. They equal, 
    respectively,
    \begin{equation}
        q_{k_i-1}c_{n_i},\ q_{k_i-1}c_{n_i-1}+q_{k_i-2}c_{n_i},\ q_{k_i-1}c_{n_i-2}+q_{k_i-2}c_{n_i-1}+q_{k_i-3}c_{n_i},\ \ldots,\ q_0c_0,
        \label{eq:survival_prob_ti}
    \end{equation}
    where $c_j \equiv \binom{n_i}{j}p^j(1-p)^{n_i-j}, j=0,1,\ldots,n_i$. Clearly, 
    $N_j$ is the coefficient of $x^j$ in 
    \begin{equation}
        \left(q_{k_i-1}x^{k_i-1} + q_{k_i-2}x^{k_i-2} + \cdots + q_0\right)\left(c_{n_i}x^{n_i} + c_{n_i-1}x^{n_i-1} + \cdots + c_0\right)
        \label{eq:polynomial_multiplication}
    \end{equation}
    for $0 \leq j \leq k_i+n_i-1$. This polynomial multiplication can be performed 
    in time $O(k_i\ \text{log}\ n_i)$ by the FFT (see~\cite{Algo}). This running time 
    simplifies to $O(n\ \text{log}\ n_i)$ as $k_i \leq n$.
 
    Now introduce the barrier $b_i$. The survival 
    probabilities~(\ref{eq:survival_prob_ti}) need to be lowered to reflect the 
    presence of the barrier. We will calculate the probability of hitting the barrier 
    for each of the $k_i+n_i$ nodes at time $t_i$ and then subtract it from that 
    node's survival probability given in~(\ref{eq:survival_prob_ti}). We explain the 
    ideas with reference to Figure~\ref{fig:algor_part_2}. First, reflect those $k_i$ 
    nodes at time $t_{i-1}+\Delta t$ over the barrier to obtain a \emph{reflected} 
    binomial structure. In Figure~\ref{fig:algor_part_2}, the reflected binomial 
    structure emanates from nodes $\text{A}_0'$, $\text{A}_1'$, and $\text{A}_2'$, 
    which start with the same survival probabilities as nodes $\text{A}_0$, 
    $\text{A}_1$, and $\text{A}_2$, respectively. Apply the same polynomial 
    multiplication~(\ref{eq:polynomial_multiplication}) to the reflected binomial 
    structure to obtain the probabilities of reaching its nodes at time $t_i$. By 
    the reflection principle, these probabilities are the desired probabilities if 
    the following adjustment is made \emph{before} the 
    multiplication~(\ref{eq:polynomial_multiplication}) commences. Consider any path 
    starting from a node above the barrier at time $t_{i-1}+\Delta t$ (one of the 
    nodes $\text{A}_0$--$\text{A}_2$). Let $\ell$ be this node's distance to barrier. 
    The up moves and down moves in the reflected path are reversed before the 
    $x$-axis is hit (recall Figure~\ref{fig:reflection_principle}). That means the 
    survival probability of each of these $k_i$ nodes in the reflected binomial 
    structure (nodes $\text{A}_0'$--$\text{A}_2'$) needs to be multiplied by 
    $[\, (1-p)/p\, ]^\ell$ before the polynomials are multiplied. The resulting 
    probabilities of the nodes above the barrier at time $t_i$ are then subtracted 
    from the corresponding probabilities in~(\ref{eq:survival_prob_ti}). This step 
    also costs $O(n\ \text{log}\ n_i)$ time with the FFT\@. 

    \begin{figure}[!t]
        \centering
        \resizebox{0.65\textwidth}{!}{
            \begin{tikzpicture}
                [very_big_circles/.list={%(10,1),(12,1),(14,1),
                (7,2),(15,2),
                (6,3),(8,3),(10,3),(12,3),(14,3),(16,3),
                (5,4),(7,4),(9,4),(11,4),(13,4),(15,4),(17,4),
                (4,5),(6,5),(8,5),(10,5),(12,5),(14,5),(16,5),(18,5),
                (3,6),(5,6),(7,6),(9,6),(11,6),(13,6),(15,6),(17,6),(19,6),
                (2,7),(4,7),(6,7),(8,7),(10,7),(12,7),(14,7),(16,7),(18,7),(20,7),
                (1,8),(3,8),(5,8),(7,8),(9,8),(11,8),(13,8),(15,8),(17,8),(19,8),(21,8),
                (18,3),(20,3),(22,3),
                (19,4),(21,4),(23,4),
                (20,5),(22,5),(24,5),
                (21,6),(23,6),(25,6),
                (22,7),(24,7),(26,7),
                (23,8),(25,8),(27,8)},
                very_big_circles/.list={(9,2),(11,2),(13,2),(17,2),(19,2),(21,2)},
                left extended/.style={shorten <=-#1},
                left extended/.default=1cm,
                right extended/.style={shorten >=-#1},
                right extended/.default=1cm]
                \matrix (tree) [nodes in empty cells,
                    matrix of nodes,
                    nodes={anchor=center,minimum size=0.75cm},
                    column sep=2cm,
                    row sep=0.25cm,
                    ampersand replacement=\&]{
                    \&  \&  \& \& \& \& \& \\
                    \&  \&  \& \& \& \& \& \\
                    \&  \&  \& \& \& \& \& \\ 
                    \&  \&  \& \& \& \& \& \\
                    \&  \&  \& \& \& \& \& \\
                    \&  \&  \& \& \& \& \& \\
                    \&  \&  \& \& \& \& \& \\
                    \&  \&  \& \& \& \& \& \\
                    \&  $\text{A}_2$ \&  \& \& \& \& \& \\
                    \&  \&  \& \& \& \& \& \\
                    \&  $\text{A}_1$ \&  \& \& \& \& \& \\
                    \&  \&  \& \& \& \& \& \\
                    \&  $\text{A}_0$ \&  \& \& \& \& \& \\
                    \&  \&  \& \& \& \& \& \\
                    \&  \&  \& \& \& \& \& \\
                    \&  \&  \& \& \& \& \& \\ 
                    \&  $\text{A}_0'$\&  \& \& \& \& \& \\
                    \&  \&  \& \& \& \& \& \\
                    \&  $\text{A}_1'$\&  \& \& \& \& \& \\
                    \&  \&  \& \& \& \& \& \\
                    \&  $\text{A}_2'$\&  \& \& \& \& \& \\
                    \&  \&  \& \& \& \& \& \\
                    \&  \&  \& \& \& \& \& \\
                    \&  \&  \& \& \& \& \& \\
                    \&  \&  \& \& \& \& \& \\
                    \&  \&  \& \& \& \& \& \\
                    \&  \&  \& \& \& \& \& \\
                    $|$  \& $|$ \& \& \& \& \& \& $|$\\
                };
            \begin{pgfonlayer}{background}
                \draw[dashed, left extended=1.7cm, right extended=1.7cm] (tree-1-8) -- (tree-27-8);
                \foreach \y in {2,...,7}{
                    \draw[dashed, left extended=1.25cm, right extended=1.25cm] (tree-1-\y) -- (tree-27-\y);}
                \foreach \x in {2,4,6,8,10,12,14,16,18,20,22,24,26}{
                    \draw[dashed, left extended=1.3cm, right extended=1.3cm] (tree-\x-2) -- (tree-\x-8);}
                \foreach \x in {7,9,11,13,15}{
                    \draw[dashed, left extended=1.7cm, right extended=1.7cm] (tree-\x-2) -- (tree-\x-8);}
                \foreach \x in {1,3,5,23,25,27}{
                    \draw[dashed, left extended=1.3cm, right extended=1.7cm] (tree-\x-2) -- (tree-\x-8);}
                \foreach \x in {17,19,21}{
                    \draw[dashed, left extended=1.9cm, right extended=1.8cm] (tree-\x-2) -- (tree-\x-8);}
                \draw[line width=1.5mm, left extended] (tree-15-1) -- (tree-15-8) node [right=1cm,font=\YUGE] {$b_i$};
                \draw[fill=black,thick,opacity=0.1] (tree-3-8.mid) -- (tree-9-2.mid) -- (tree-13-2.mid) -- (tree-19-8.mid) -- cycle ;
                \draw[fill=black,thick,opacity=0.5] (tree-11-8.mid) -- (tree-17-2.mid) -- (tree-21-2.mid) -- (tree-27-8.mid) -- cycle ;
            \end{pgfonlayer}
            \draw[-] (tree-28-1.mid) -- (tree-28-8.mid);
            \node[below=0.5cm,font=\YUGE] at (tree-28-1) {$t_{i-1}$};
            \node[below=0.5cm,font=\YUGE] at (tree-28-2.east) {$t_{i-1}+\Delta t$};
            \node[below=0.5cm,font=\YUGE] at (tree-28-8) {$t_{i}$};
        \end{tikzpicture}}
        \caption{The reflected binomial structure. Above, the 
                 light gray area contains the binomial structure 
                 emanating from the $k_i$ nodes 
                 above the barrier at time $t_{i-1}+\Delta t$.
                 Here, $k_i = 3$. The dark gray area contains the 
                 binomial structure emanating from these
                 $k_i$ nodes' reflected counterparts. The 
                 branches of the binomial structures are not 
                 shown.}
        \label{fig:algor_part_2}
    \end{figure}

    In summary, each iteration of part (2) takes time
    \begin{equation*}
        O\left(n\ \text{log}\ n_i + n\ \text{log}\ n_i\right) = O\left(n\ \text{log}\ n_i\right) = O\left(n\ \text{log}\ \frac{n}{m}\right).
    \end{equation*}
    With $m$ iterations in total, part (2) takes time 
    \begin{equation*}
        O\left(mn\ \text{log}\ \frac{n}{m}\right).
    \end{equation*}
    The pseudo code appears in Algorithm~\ref{alg:core}.

    \begin{algorithm}
        \caption{The core algorithm}\label{alg:core}
        \textbf{Input}: $\mu$, $\sigma$, $p_1, p_2, \ldots, p_m$, $t_0, t_1, t_2, \ldots, t_m$, $n$. \\
        \textbf{Output}: The implied barrier levels $b_1,b_2,\ldots,b_m$ on the tree that reproduce $p_1,p_2,\ldots,p_m$.
        \begin{algorithmic}[1]
            \State $\Delta t \gets t_m/n$
            \State $t_i \gets \Delta t \times \text{round}(t_i/\Delta t)$ for $i=1,2,\ldots,m-1$
            \For{$i \gets 1$ to $m$}
                \State $n_i \gets [\,(t_i - t_{i-1})/\Delta t\,] - 1$                          \Comment{Assume $n_i$ is even below for brevity}
                \State Position the initial lattice with $\beta=-\sigma\sqrt{\Delta t}$   \Comment{Part (1) starts here}
                \State $\text{B}_\ell \gets \sum_{j=0}^{\frac{n_i+\ell}{2}-1}\binom{n_i}{j}p^{n_i-j}(1-p)^j-\sum_{j=\ell}^{\frac{n_i+\ell}{2}-1}\binom{n_i}{j-\ell}p^{n_i-j}(1-p)^{j}$ for $\ell=2,4,\ldots,n_i$
                \State $\text{B}_0 \gets 0$                                     \Comment{$\text{B}_\ell=1$ for $\ell > n_i$ if accessed}
                \State L $\gets$ the lowest node at time $t_i$ reachable from nodes above $b_{i-1}$ at time $t_{i-1}$
                \State U $\gets$ the upmost node at time $t_i$ reachable from nodes above $b_{i-1}$ at time $t_{i-1}$
                \State $v_\text{L} \gets 1$ ; $v_\text{U} \gets 0$
                \While{L and U are not adjacent}
                    \State M $\gets$ the median node at time $t_i$ between L and U
                    \State $k_i \gets$ the number of nodes above M at time $t_{i-1}+\Delta t$ 
                    \State Let $q_{k_i-1}, q_{k_i-2}, \ldots, q_0$ be the survival probabilities of these $k_i$ nodes from the top down
                    \State Let $\ell_{k_i-1}, \ell_{k_i-2}, \ldots, \ell_0$ be these nodes' distances to barrier M 
                    \State $v_\text{M} \gets \sum_{j=0}^{k_i-1}q_jB_{\ell_j}$
                    \If{ $v_\text{U} < 1-p_i$ and $v_\text{M} \geq 1-p_i$}
                        \State L $\gets$ M ; $v_\text{L} \gets v_\text{M}$
                    \Else
                        \State U $\gets$ M ; $v_\text{U} \gets v_\text{M}$
                    \EndIf
                \EndWhile
                \State Solve the quadratic function of $\beta$ for the survival probability that equates $1-p_i$
                \State $b_i \gets$ $e^{\beta}\ \times\,$ the stock price of node L
                \State $k_i \gets$ the number of nodes above $b_i$ at time $t_{i-1}+\Delta t$  \Comment{Part (2) starts here}
                \State Let $q_{k_i-1}, q_{k_i-2}, \ldots, q_0$ be the survival probabilities of these $k_i$ nodes from the top down
                \State Let $\ell_{k_i-1}, \ell_{k_i-2}, \ldots, \ell_0$ be these nodes' distances to barrier $b_i$
                \State $c_j \gets \binom{n_i}{j}p^j(1-p)^{n_i-j}$ for $j=0,1,\ldots,n_i$
                \State Let $\sum_{j=0}^{k_i+n_i-1}N_{j}x^j = \left(\sum_{j=0}^{k_i-1}q_jx^j\right)\left(\sum_{j=0}^{n_i}c_jx^j\right)$ 
                \State $s_j \gets \left[\,(1-p)/p\,\right]^{\ell_{k_i-1}} q_j$ for $j=0,1,\ldots,k_i-1$
                \State Let $\sum_{j=0}^{k_i+n_i-1}N_{j}'x^j = \left(\sum_{j=0}^{k_i-1}s_{k_i-1-j}x^j\right)\left(\sum_{j=0}^{n_i}c_jx^j\right)$ 
                \State $N_j \gets N_j - N_{k_i+\ell_0+j-1}'$ for $j=0,1,\ldots,n_i-\ell_0$
            \EndFor
        \end{algorithmic}
    \end{algorithm}

\section{Adaptations of the Core Algorithm}\label{sec:algorithms-adaptations}

    Although the core algorithm is to find the implied step barrier, it can be easily 
    specialized to handle other kinds of moving barriers. Moreover, the resulting 
    algorithms become pricing algorithms for options with such barriers after 
    switching to the risk-neutral measure and making minor changes. The above results 
    continue to hold when the growth rate and volatility are time varying.

\subsection{Barrier Options Pricing}\label{subsec:algorithms-adaptation-pricing}

    To price step-barrier options, drop part (1) of the core algorithm, which is
    about finding the implied barrier, thus irrelevant here. Instead, run part (2) to
    calculate the survival probabilities for the terminal nodes, but under the
    measure $Q$, meaning $\mu = r$. Then evaluate the 
    expectation~(\ref{eq:risk_neutral}) by summing the terminal nodes' survival 
    probability-weighted payoffs before discounting the sum, which only takes $O(n)$ 
    time. In total, the running time equals that of part (2) of the core algorithm, 
    $O(mn\ \text{log}\ (n/m))$. This $O(n/\text{log}n)$ improvement over the standard 
    algorithm's $O(n^2)$ is dramatic. 

\subsection{Barriers: The Discrete Case}\label{subsec:algorithms-adaptation-discrete}

    Discrete barriers are monitored at discrete time points. The barrier levels need 
    not be identical. The core algorithm can be simplified to find the implied 
    discrete barrier as follows. Because there is no barrier within time 
    interval $(\, t_{i-1}, t_i \,)$ any more, the reflection principle is not needed, 
    and formula~(\ref{eq:survival_probability}) is reduced to its first term $G_1$. 
    As a result, formula~(\ref{eq:survival_probability_rec}) simplifies to
    \begin{equation*}
        B_{\ell-2} \quad\equiv\quad G_1 - \binom{n_i}{\frac{n_i+\ell}{2}-1}p^{\frac{n_i-\ell}{2}+1}(1-p)^{\frac{n_i+\ell}{2}-1}.
    \end{equation*}
    Part (2) now stops at formula~(\ref{eq:polynomial_multiplication}), omitting the 
    subsequent subtractions to account for the barrier \emph{within} the time 
    interval. The running time therefore remains the same at $O(mn\ \text{log}\ n)$. 
    To price discrete-barrier options, simply follow the idea at the end of
    Subsection~\ref{subsec:algorithms-adaptation-pricing} with the same running time 
    $O(mn\ \text{log}\ (n/m))$.

\subsection{Barriers: The General Case}\label{subsec:algorithms-adaptation-general}
    
    Suppose the moving barrier alternates between a time interval with a continuously 
    monitored constant barrier level and a time interval with a discretely monitored 
    barrier at the end of it. This setup subsumes the partial and window barriers 
    (see Figure~\ref{fig:barriers}). To find the implied barrier, simply alternate 
    between the core algorithm and the algorithm in 
    Subsection~\ref{subsec:algorithms-adaptation-discrete} with the same running time 
    $O(mn\ \text{log}\ n)$. Similarly, such options can be priced in 
    $O(mn\ \text{log}\ (n/m))$ time.

    For a continuously monitored general moving barrier $b(t)$, the algorithm to find 
    the implied barrier again adopts the core algorithm. But it grows only the 
    trinomial structure of the BTT out of every tree nodes, without the ensuing 
    binomial part (hence $n_i = 0$ for all $i$). This results in a trinomial tree. 
    The running times are $O(n^2\ \text{log}\ n)$ for finding the implied barrier --- 
    because the tree has $O(n^2)$ nodes and the bisection method needs 
    $O(\text{log}\ n)$ iterations --- and $O(n^2)$ for pricing the moving-barrier 
    options.

\subsection{Time-Varying Models}\label{subsec:algorithms-adaptation-tvaring}

    So far, our algorithms assumed a constant growth rate $\mu$ and volatility 
    $\sigma$. When the growth rate $\mu(t)$ and volatility $\sigma(t) > 0$ are time 
    varying, the algorithms for handling continuously monitored moving 
    barriers are the same except for one simple change: The spacing between two 
    vertically adjacent nodes is set to $\sigma'\sqrt{\Delta t}$, where $\sigma'$ is 
    any upper bound on the volatility function between times $t_0$ and $t_m$. As a 
    result, for the nodes residing at time $t$, 
    formulas~(\ref{eq:alpha})--(\ref{eq:gamma}) become
    \begin{align*}
        \alpha\quad &\equiv\quad \beta + \sigma' \sqrt{\Delta t}, \\
        \beta\quad &\equiv\quad \hat{\theta} - \theta(t), \\
        \gamma\quad &\equiv\quad \beta - \sigma' \sqrt{\Delta t},
    \end{align*}
    where $\theta(t) \equiv [\,\mu(t)-\sigma^2(t)/2\,]\,\Delta t$. 
    Equation~(\ref{eq:variance}) becomes
    \begin{equation*}
        P_u\alpha^2 + P_m\beta^2 + P_d\gamma^2 = \sigma^2(t)\,\Delta t.
    \end{equation*}
    The branching probabilities are guaranteed to be valid for $\Delta t$ suitably 
    small (see~\cite{Lok2020}). For pricing, $\theta(t)$ is replaced by $r(t)$, the
    time-varying risk-free interest rate. The running time is again 
    $O(n^2\ \text{log}\ n)$ for finding the exact implied general moving barrier and 
    $O(n^2)$ for pricing moving-barrier options.

\section{Numerical Results}\label{sec:algorithms-experiments}

    This section provides the numerical results on obtaining the exact implied 
    barriers and pricing barrier options for various barrier types. We consider 
    constant barriers, step barriers, discrete barriers, and partial barriers. 
    Running times are based on an Amazon EC2 c6g.medium instance. All programs are 
    written in Python using the NumPy package.

\subsection{Exact Implied Barriers}
        
    The core algorithm produces an exact implied barrier for all $n$. Although 
    different $n$s yield different implied barriers, their differences from the true 
    barrier are minuscule even for very small $n$s in typical cases. The initial 
    stock price is one dollar in this subsection. We first illustrate this with the 
    constant-barrier case, which is treated as a subcase of the step barrier. Given 
    a constant barrier level, the barrier-hitting probability has a well-known 
    formula (see~\cite{Ingersoll1987}). The solid lines in Figure~\ref{fig:exact_ib} 
    plot the exact implied barriers found by the core algorithm for five different 
    true barrier levels (dotted lines), unknown to the algorithm of course. The exact 
    implied barriers converge quickly and monotonically to the true barrier as $n$ 
    increases. Take the 3rd exhibit for example. The true barrier level is 0.85, 
    and the resulting barrier-hitting probability is 0.424910. The absolute errors 
    start at 0.014 (or $1.647\%$ of the true barrier) for $n = 1$ and decrease 
    smoothly toward zero as $n$ increases. They are less than 0.001 (or $0.133\%$ of 
    the true barrier) for $n \geq 43$. The dashed lines with markers in 
    Figure~\ref{fig:exact_ib} reveal that extrapolating the implied barrier levels 
    from those of $n$ and $2n+1$ reduces the errors even further 
    (see~\cite{Atkinson1989}). Now they start at 0.003 (or within $0.353\%$ of the 
    true barrier) using $n = 1$ and $n = 3$ and never exceed 0.001 (or within 
    $0.118\%$ of the true barrier) when $n \geq 5$. 
    
    \begin{figure}[!t]
        \centering
        \includegraphics[width = \textwidth]{figures/exact_implied_barrier_0.9999.pdf}
        \includegraphics[width = \textwidth]{figures/exact_implied_barrier_0.95.pdf}
        \includegraphics[width = \textwidth]{figures/exact_implied_barrier_0.85.pdf}
        \includegraphics[width = \textwidth]{figures/exact_implied_barrier_0.75.pdf}
        \includegraphics[width = \textwidth]{figures/exact_implied_barrier_0.35.pdf}
        \caption{The solid lines are the implied barriers from the core algorithm 
                 for $n = 1,3,5,\ldots,101$ at five different true barrier levels 
                 (dotted lines). The current stock price is $1$ dollar, the growth 
                 rate is $10\%$, the volatility is $25\%$, the horizon is $1$ year. 
                 The dashed lines with markers are the extrapolated implied barriers 
                 using $n$ and $2n+1$.}
         \label{fig:exact_ib}
    \end{figure}

    We make two observations about Figure~\ref{fig:exact_ib}. First, in the 
    continuous-time model, the barrier must stay below the current stock price for 
    any non-zero survival probability. But our exact implied barrier might violate 
    that when the true barrier is lower than, but very close to, the current stock 
    price, in which case the barrier-hitting probability is very close to $1$. For 
    example, the first exhibit depicts such a barrier-too-close situation to the 
    extreme: The true barrier level 0.9999 is merely 0.0001 below the current stock 
    price 1. The exact implied barrier is found to be 1.355 with a huge relative 
    error of $35.514\%$. But it quickly drops to 1.171 for $n = 3$ and 1.079 after 
    extrapolation using $n = 1$ and $n = 3$. Second, when the true barrier is so low 
    that the resulting barrier-hitting probability is too negligible to encounter in 
    reality, our exact implied barrier may deviate significantly from the true 
    barrier at small $n$s. For example, the implied barrier is far from the true 
    barrier level $0.35$ for small $n$s in the fifth exhibit, where the 
    barrier-hitting probability is \num{8.15219e-6}. To address it, one can raise $n$ 
    suitably or apply extrapolation. For example, the relative error is only 
    $1.790\%$ for $n = 51$ and $0.309\%$ after extrapolation using $n = 25$ and 
    $n = 51$. For typical barrier-hitting probabilities such as the middle three 
    exhibits, even $n = 1$ works surprisingly well. We remark that extrapolation will 
    not be applied for the remaining experiments unless explicitly stated.
    
    The literature does not offer true step-barrier option prices for the interesting 
    case of large numbers of barrier levels. We therefore consider $m = 60$ 
    \emph{identical} month-long barrier levels for a step barrier. The $m$ 
    barrier-hitting probabilities are generated from the well-known formula. Finally, 
    we run the core algorithm, which does not know it is dealt a constant-barrier 
    case, on these probabilities to obtain the exact implied step barrier. The result 
    appears in Figure~\ref{fig:step_barrier_convergence}. Both the left and right 
    panels plot the exact implied barrier for $n = 180, 780, 1980, 6060$. The true 
    barrier is 0.85 for 5 years in the left panel and 0.9999 for 5 years in the right 
    panel. In both settings, the exact implied barrier converges to the true barrier 
    as $n$ increases. 
    
    \begin{figure}[!t]
    \centering
        \includegraphics[width = 0.485\textwidth]{figures/experiment_step_barrier_convergence_85.pdf}
        %\includegraphics[width = 0.505\textwidth]{figures/experiment_maximum_error_0.85.pdf}
        \includegraphics[width = 0.485\textwidth]{figures/experiment_step_barrier_convergence_9999.pdf}
        %\includegraphics[width = 0.505\textwidth]{figures/experiment_maximum_error_0.9999.pdf}
        \caption{The exact implied step barriers by the core algorithm.
                 %and their 
                 %maximum errors. The maximum errors are plotted in log scale. 
                 The parameters are the same as those in Figure~\ref{fig:exact_ib}. 
                 The true barriers for the left and right exhibits are 0.85 and 0.9999, 
                 respectively.}
        \label{fig:step_barrier_convergence}
    \end{figure}

    We next apply the core algorithm to real-world data. The term structures of 
    default probabilities (our cumulative barrier-hitting probabilities $p_i$) up to 
    60 months, annual growth rates ($\mu$), and volatilities of asset returns 
    ($\sigma$) are provided by the Credit Research Initiative (CRI) of the National 
    University of Singapore.\footnote{The CRI Dataset, the Credit Research 
    Initiative of the National University of Singapore. We access the dataset on
    08-17-2023.} The left exhibit of Figure~\ref{fig:pd_ib} plots the default 
    probabilities for Freddie Mac and Fannie Mae as of February 2008. The $x$-axis 
    represents the number of forward months, and the $y$-axis denotes the default 
    probability. For example, Freddie Mac (Fannie Mae) has a 40\% (8\%, respectively) 
    chance of defaulting within the next 10 months. The annual growth rate and 
    volatility of the asset return of Freddie Mac (Fannie Mae) are $-$1.91\% (0.68\%, 
    respectively) and 53.72\% (9.63\%, respectively).
    
    \begin{figure}[!t]
    \centering
        \includegraphics[width = 0.495\textwidth]{figures/experiment_default_prob.pdf}
        \includegraphics[width = 0.495\textwidth]{figures/experiment_implied_barrier.pdf}
        \caption{Term structures of default probabilities and exact implied step barriers 
                 of Freddie Mac and Fannie Mae as of February 2008. The implied step 
                 barriers are calculated for $n = 1500$.} 
        \label{fig:pd_ib}
    \end{figure}
    
    The right exhibit of Figure~\ref{fig:pd_ib} plots the implied step barriers found 
    by the core algorithm. Both are downward sloping. Table~\ref{tb:ib_time_24} 
    tabulates the running times of finding the implied step barrier of Freddie Mac by 
    the core algorithm and the standard algorithm for $780 \leq n \leq 36060$ (in 
    increments of 120). Note that $m = 60$ with the CRI dataset and $n_i = n/m$ for 
    $i = 1,2,\ldots,60$. The running times of the core algorithm never exceed $11$ 
    seconds. They are also indistinguishable from a linear growth. 
    %(Chapter~\ref{chap:convergence} will prove that the implied constant barrier 
    %converges at a rate of $O(n^{-1})$.) 
    On the other hand, the standard algorithm takes 3.11 seconds at $n = 780$ and 
    more than two thousands seconds at $n = 36060$. Speedup is defined as the ratio 
    of the running of our core algorithm to that of the standard algorithm 
    (see~\cite{Jaja1992}). The speedups range from 8.59 to 237.39. The core algorithm 
    is obviously much more viable for large portfolios and datasets.

    \begin{table}[!t]
        \centering
        \caption{Running times (in seconds) of finding the implied barrier of Freddie 
                 Mac by the core algorithm and speedups over the standard algorithm.}
        \begin{tabular}{rrrr}
        \specialrule{.1em}{.05em}{.05em}
            $n$   & Core  algorithm & Standard  algorithm & Speedup \\ \midrule
            780   & 0.36           &               3.11  &   8.59  \\
            1500  & 0.62           &               7.76  &  12.55  \\
            2340  & 0.89           &              15.90  &  17.87  \\
            3180  & 1.15           &              26.61  &  23.19  \\
            4500  & 1.50           &              49.78  &  33.10  \\
            6060  & 2.02           &              82.84  &  40.92  \\
            9060  & 3.02           &             179.33  &  59.30  \\
            12060 & 3.82           &             307.59  &  80.51  \\
            18060 & 5.56           &             663.41  & 119.29  \\
            24060 & 7.09           &            1160.80  & 163.66  \\ 
            30060 & 9.02           &            1784.62  & 197.81  \\           
            36060 & 10.92          &            2592.52  & 237.39  \\ \specialrule{.1em}{.05em}{.05em}          
        \end{tabular}
        \label{tb:ib_time_24}
    \end{table}
        
\subsection{Barrier Options Pricing}
    
    Our pricing algorithms are fast and have excellent convergence. 
    Figure~\ref{fig:single_barrier_option_pricing} confirms it with a 
    constant-barrier DOC. As $n$ increases from 201 to 1,401 by increments of 2, the 
    prices converge to 5.9968 given by Merton's (1974) formula. The absolute errors 
    are at most 0.001. Figure~\ref{fig:step_barrier_option_pricing} covers the case 
    of a step-barrier DOC. The unbiased and highly efficient 
    Brownian-bridge method of \cite{Shevchenko2003} with 10 million paths gives a 
    mean of 6.241 and a standard error of 0.004. Our algorithm's prices never deviate 
    from that by more than 0.02. The speedups over the standard algorithm range 
    from 11.55 to 48.67 for $603 \leq n \leq 4203$ (in increments of 6). 
    %Recall that the standard algorithm works on a tree node by node. The running time 
    %improvement, $O(n/\text{log}n)$, is practically indistinguishable from $O(n)$. 
        
    \begin{figure}[!t]
        \centering
        \includegraphics[width = 0.6\textwidth]{figures/experiment_single_barrier_option_pricing.pdf}
        \caption{Convergence in pricing a constant-barrier DOC by our algorithm. The 
                 initial stock price is 95, the strike price is 100, the 
                 risk-free interest rate is $10\%$, the volatility is $25\%$, the 
                 time to maturity is 1 year, and the barrier is 90. The true price 
                 is 5.9968 (dotted line).}
        \label{fig:single_barrier_option_pricing}
    \end{figure}
    
    \begin{figure}[!t]
    \centering
        \includegraphics[width = 0.495\textwidth]{figures/experiment_step_barrier_option_pricing.pdf}
        \includegraphics[width = 0.495\textwidth]{figures/experiment_step_barrier_option_pricing_speedup.pdf}
        \caption{Convergence and speedups in pricing a step-barrier DOC by our 
                 algorithm. The barrier function is 90 followed by 80 followed 
                 by 90 with equal durations. The rest of the parameters are the 
                 same as those in Figure~\ref{fig:single_barrier_option_pricing}.
                 The left exhibit plots the convergence of our algorithm. The 
                 Brownian-bridge method produces a very tight 95\% confidence 
                 interval (CI) of $[\, 6.233, 6.249 \,]$ (dotted lines). The right 
                 exhibit plots the speedups of our algorithm over the standard 
                 algorithm.}
        \label{fig:step_barrier_option_pricing}
    \end{figure}

    We next price discrete DOCs at different constant barrier levels with an initial 
    stock price of 100. Our algorithm with extrapolation is compared with the 
    approximation formula of \cite{Broadie1997}. When $m = 50$ monitoring points, our 
    relative errors are never more than $2.147\%$. The formula in comparison produces 
    smaller relative errors than ours except when the barrier is 99 (see 
    Table~\ref{tb:discrete_price}). Figure~\ref{fig:discrete_barrier_option_pricing}
    illustrates the convergence and speedups of our algorithm over the standard 
    algorithm. The speedups trend roughly from 20 to 70 for $5250 \leq n \leq 25250$. 
    Table~\ref{tb:discrete_price2} provides the option prices with 
    $m = 5\ \text{and}\ 25$. The absolute relative errors of our algorithm stay below 
    $2.170\%$ whereas the formula's reach $4.977\%$ when $m = 25$ and $9.779\%$ when 
    $m = 5$. In general, the formula's performance seems to degrade as the number of 
    monitoring points decreases or as the barrier approaches the initial stock price. 
    Table~\ref{tb:discrete_price3} compares the results under different volatilities, 
    times to maturities, and strike prices. As before, the formula's relative errors 
    jump upward when the barrier is close to the initial stock price. Its relative 
    errors also tend to deteriorate when the volatility or the time to maturity 
    increases. In contrast, our algorithm is less sensitive to such parameter 
    changes. Finally, unlike our algorithm, the formula only works for constant 
    barrier levels.

    \begin{table}[!t]
    \centering
        \caption{Comparison of discrete DOC prices with $m = 50$ monitoring points. 
                 The initial stock price is $100$, the strike price is $100$, the 
                 risk-free interest rate is $10\%$, the volatility is $30\%$, the 
                 time to maturity is $0.2$ year, and the barrier levels are the same 
                 at all the 50 equally spaced monitoring points, which approximates 
                 daily monitoring. The 2nd, 4th, and 5th columns are from Table 2.1 
                 of \cite{Broadie1997}. The 3rd column extrapolates the prices by 
                 our algorithm using $n = 12550$ and $n = 25050$.}
        \begin{tabular}{rrrrrr}
            \specialrule{.1em}{.05em}{.05em} 
            \multicolumn{1}{c}{\multirow{2}{*}{\begin{tabular}[c]{@{}c@{}}Barrier\\ level\end{tabular}}} & \multicolumn{1}{c}{\multirow{2}{*}{\begin{tabular}[c]{@{}c@{}}(a) Broadie et al.\\ (1997)\end{tabular}}} & \multicolumn{1}{c}{\multirow{2}{*}{\begin{tabular}[c]{@{}c@{}}(b) Our algorithm\\ with extrapolation\end{tabular}}} & \multicolumn{1}{c}{\multirow{2}{*}{True}} & \multicolumn{2}{c}{Relative error (\%)} \\ \cline{5-6} 
            \multicolumn{1}{c}{}                         & \multicolumn{1}{c}{}                                                                                                                   & \multicolumn{1}{c}{}                                                                                              & \multicolumn{1}{c}{}                      & \multicolumn{1}{c}{(a)}                     & \multicolumn{1}{c}{(b)}                     \\ \midrule
            %90                                           & 6.098                                                                                                                                  & 6.092                                                                                                             & 6.098                                     & 0.0                                         & $-0.1$                                        \\
            91                                           & 5.977                                                                                                                                  & 5.969                                                                                                             & 5.977                                     & 0.000                                         & $-0.133$                                        \\
            92                                           & 5.810                                                                                                                                  & 5.799                                                                                                             & 5.810                                     & 0.000                                         & $-0.195$                                        \\
            93                                           & 5.585                                                                                                                                  & 5.569                                                                                                             & 5.584                                     & 0.018                                         & $-0.262$                                        \\
            94                                           & 5.288                                                                                                                                  & 5.268                                                                                                             & 5.288                                     & 0.000                                         & $-0.372$                                        \\
            95                                           & 4.907                                                                                                                                  & 4.882                                                                                                             & 4.907                                     & 0.000                                         & $-0.519$                                        \\
            96                                           & 4.428                                                                                                                                  & 4.395                                                                                                             & 4.427                                     & 0.023                                         & $-0.728$                                        \\
            97                                           & 3.836                                                                                                                                  & 3.795                                                                                                             & 3.834                                     & 0.052                                         & $-1.025$                                        \\
            98                                           & 3.121                                                                                                                                  & 3.080                                                                                                             & 3.126                                     & $-0.160$                                      & $-1.466$                                        \\
            99                                           & 2.271                                                                                                                                  & 2.287                                                                                                             & 2.337                                     & $-2.824$                                      & $-2.147$                                        \\ \specialrule{.1em}{.05em}{.05em}
            %99.1                                         & 2.179                                                                                                                                  & 2.206                                                                                                             & 2.182                                     & $-0.137$                                      & $1.100$                                        \\ 
            %99.2                                         & 2.084                                                                                                                                  & 2.126                                                                                                             & 2.021                                     & $3.117$                                       & $5.195$                                        \\ 
            %99.3                                         & 1.989                                                                                                                                  & 2.046                                                                                                             & 2.014                                     & $-1.241$                                      & $1.589$                                        \\ 
            %99.4                                         & 1.892                                                                                                                                  & 1.966                                                                                                             & 1.900                                     & $-0.421$                                      & $3.474$                                        \\ 
            %99.5                                         & 1.793                                                                                                                                 & 1.887                                                                                                             & 1.852                                     & $-3.186$                                       & $1.890$                                        \\ \specialrule{.1em}{.05em}{.05em} 
            \end{tabular}
        \label{tb:discrete_price}
    \end{table}

    \begin{figure}[!t]
    \centering
        \includegraphics[width = 0.495\textwidth]{figures/experiment_option_pricing_discrete91.pdf}
        \includegraphics[width = 0.495\textwidth]{figures/experiment_option_pricing_discrete91_time.pdf}
        \includegraphics[width = 0.495\textwidth]{figures/experiment_option_pricing_discrete99.pdf}
        \includegraphics[width = 0.495\textwidth]{figures/experiment_option_pricing_discrete99_time.pdf}
        \caption{Convergence and speedups in pricing discrete DOCs by our algorithm. 
                 The parameters are the same as those in 
                 Table~\ref{tb:discrete_price}. The left exhibits show the 
                 convergence of our algorithm. The right exhibits plot the speedups 
                 of our algorithm over the standard algorithm. The $x$-axes denote 
                 the number of time steps.} 
        \label{fig:discrete_barrier_option_pricing}
    \end{figure}

    \begin{table}[!t]
    \centering
        \caption{Comparison of discrete DOC prices with $m = 5,\ 25$ monitoring points. 
                 The parameters are the same as those in Table~\ref{tb:discrete_price} 
                 except $m$. The 3nd, 5th, and 6th columns are from Table 2.2 of 
                 \cite{Broadie1997}. The 4th column extrapolates the prices by our 
                 algorithm using $n = 6275$ and $n = 12525$ for $m = 25$, and 
                 $n = 1255$ and $n =  2505$ for $m = 5$.}
        \begin{tabular}{lrrrrrr}
        \specialrule{.1em}{.05em}{.05em}
            \multirow{2}{*}{$m$}  & \multicolumn{1}{c}{\multirow{2}{*}{\begin{tabular}[c]{@{}c@{}}Barrier\\ level\end{tabular}}} & \multicolumn{1}{c}{\multirow{2}{*}{\begin{tabular}[c]{@{}c@{}}(a) Broadie et al.\\ (1997)\end{tabular}}} & \multicolumn{1}{c}{\multirow{2}{*}{\begin{tabular}[c]{@{}c@{}}(b) Our algorithm\\ with extrapolation\end{tabular}}} & \multicolumn{1}{c}{\multirow{2}{*}{True}} & \multicolumn{2}{c}{Relative error (\%)} \\ \cline{6-7} 
                            & \multicolumn{1}{c}{}                         & \multicolumn{1}{c}{}                                                                                                                   & \multicolumn{1}{c}{}                                                                                              & \multicolumn{1}{c}{}                      & \multicolumn{1}{c}{(a)}                     & \multicolumn{1}{c}{(b)}                     \\ \midrule
        \multirow{5}{*}{25} & 91                                           & 6.033                                                                                                                                  & 6.022                                                                                                             & 6.032                                     & 0.017                                         & $-0.161$                                        \\
                            & 93                                           & 5.688                                                                                                                                  & 5.669                                                                                                             & 5.688                                     & 0.000                                         & $-0.337$                                        \\
                            & 95                                           & 5.084                                                                                                                                  & 5.049                                                                                                             & 5.081                                     & 0.059                                         & $-0.624$                                        \\
                            & 97                                           & 4.113                                                                                                                                  & 4.067                                                                                                             & 4.116                                     & $-0.073$                                      & $-1.202$                                        \\
                            & 99                                           & 2.673                                                                                                                                  & 2.752                                                                                                             & 2.813                                     & $-4.977$                                      & $-2.170$                                        \\ \midrule
        \multirow{5}{*}{5}  & 91                                           & 6.194                                                                                                                                  & 6.175                                                                                                             & 6.187                                     & 0.113                                         & $-0.187$                                        \\
                            & 93                                           & 6.004                                                                                                                                  & 5.977                                                                                                             & 6.000                                     & 0.067                                         & $-0.385$                                        \\
                            & 95                                           & 5.646                                                                                                                                  & 5.633                                                                                                             & 5.671                                     & $-0.441$                                      & $-0.672$                                        \\
                            & 97                                           & 5.028                                                                                                                                  & 5.111                                                                                                             & 5.167                                     & $-2.690$                                      & $-1.080$                                        \\
                            & 99                                           & 4.050                                                                                                                                  & 4.469                                                                                                             & 4.489                                     & $-9.779$                                      & $-0.435$                                        \\ \specialrule{.1em}{.05em}{.05em}
        \end{tabular}
        \label{tb:discrete_price2}
    \end{table}

    \begin{table}[!t]
    \centering
        \caption{Comparison of discrete DOC prices under different volatilities, 
                 times to maturities, and strike prices. The parameters are the same 
                 as in Table~\ref{tb:discrete_price} except that Panel A raises the 
                 volatility to $60\%$, Panel B raises the time to maturity to $2$ 
                 years, and Panel C raises the strike price to $110$. The 3rd, 5th, 
                 and 6th columns are from Table 2.3 of \cite{Broadie1997}. The 4th 
                 column extrapolates the prices by our algorithm with 
                 $n = 12550$ and $n = 25050$.}
        \begin{tabular}{lrrrrrr}
        \specialrule{.1em}{.05em}{.05em}
        \multirow{2}{*}{Panel} & \multicolumn{1}{c}{\multirow{2}{*}{\begin{tabular}[c]{@{}c@{}}Barrier\\ level\end{tabular}}} & \multicolumn{1}{c}{\multirow{2}{*}{\begin{tabular}[c]{@{}c@{}}(a) Broadie et al. \\ (1997)\end{tabular}}} & \multicolumn{1}{c}{\multirow{2}{*}{\begin{tabular}[c]{@{}c@{}}(b) Our algorithm\\ with extrapolation\end{tabular}}} & \multicolumn{1}{c}{\multirow{2}{*}{True}} & \multicolumn{2}{c}{Relative error (\%)}            \\ \cline{6-7} 
                               & \multicolumn{1}{c}{}                         & \multicolumn{1}{c}{}                                                                                      & \multicolumn{1}{c}{}                                                                                                & \multicolumn{1}{c}{}                      & \multicolumn{1}{c}{(a)} & \multicolumn{1}{c}{(b)} \\ \midrule
        \multirow{5}{*}{A}     & 91                                           & 8.573                                                                                                     & 8.521                                                                                                               & 8.572                                     & 0.012                    & $-0.596$                   \\
                               & 93                                           & 7.566                                                                                                     & 7.499                                                                                                               & 7.563                                     & 0.040                    & $-0.845$                   \\
                               & 95                                           & 6.346                                                                                                     & 6.266                                                                                                               & 6.344                                     & 0.032                    & $-1.234$                   \\
                               & 97                                           & 4.900                                                                                                     & 4.853                                                                                                               & 4.941                                     & $-0.830$                 & $-1.771$                   \\
                               & 99                                           & 3.219                                                                                                     & 3.286                                                                                                               & 3.475                                     & $-7.367$                 & $-2.549$                   \\ \midrule
        \multirow{5}{*}{B}     & 91                                           & 16.446                                                                                                    & 16.275                                                                                                              & 16.436                                    & 0.061                    & $-0.978$                   \\
                               & 93                                           & 14.534                                                                                                    & 14.354                                                                                                              & 14.537                                    & 0.021                    & $-1.258$                   \\
                               & 95                                           & 12.371                                                                                                    & 12.250                                                                                                              & 12.451                                    & -0.643                   & $-1.617$                   \\
                               & 97                                           & 9.945                                                                                                     & 10.043                                                                                                              & 10.254                                    & $-3.013$                 & $-2.061$                   \\
                               & 99                                           & 7.243                                                                                                     & 7.950                                                                                                               & 8.061                                     & $-10.148$                & $-1.377$                   \\ \midrule
        \multirow{5}{*}{C}     & 91                                           & 2.433                                                                                                     & 2.431                                                                                                               & 2.433                                     & 0.000                    & $-0.078$                    \\
                               & 93                                           & 2.336                                                                                                     & 2.332                                                                                                               & 2.336                                     & 0.000                    & $-0.158$                    \\
                               & 95                                           & 2.136                                                                                                     & 2.127                                                                                                               & 2.135                                     & 0.047                    & $-0.358$                    \\
                               & 97                                           & 1.757                                                                                                     & 1.741                                                                                                               & 1.756                                     & 0.057                    & $-0.864$                   \\
                               & 99                                           & 1.105                                                                                                     & 1.114                                                                                                               & 1.136                                     & $-2.729$                 & $-1.910$                   \\ \specialrule{.1em}{.05em}{.05em}
        \end{tabular}
        \label{tb:discrete_price3}
    \end{table}

    \begin{figure}[!t]
    \centering
        \begin{tikzpicture}
            \begin{axis}[
                    width=0.6\textwidth,
                    height=0.45\textwidth,
                    xtick = {0.5,1.5,2.5},
                    xticklabels = {0,0.5,1},
                    xticklabel style={text height=1.5ex, font=\large},
                    ytick = {2},
                    yticklabels = {100},
                    yticklabel style={font=\large},
                    xmin=-0.5,xmax=3,
                    ymin=0,ymax=4,
                ]
                \addplot[line,-,domain=1.5:2.5]{2}; %node [midway,above] {$b_2$};
                \addplot[solid] coordinates{(1.5,2)(2.5,2)};
            \end{axis}
        \end{tikzpicture}
        \caption{A partial barrier.}
        \label{fig:partial_barrier}
    \end{figure}

    \begin{figure}[!t]
    \centering
        \includegraphics[width = 0.495\textwidth]{figures/experiment_partial_barrier_option_pricing.pdf}
        \includegraphics[width = 0.495\textwidth]{figures/experiment_partial_barrier_option_pricing_speedup.pdf}
        \caption{Convergence and speedups in pricing a partial-barrier DOC by our 
                 algorithm. The initial stock price is $105$, the strike price is 
                 $110$, the interest rate is $10\%$, the volatility is $25\%$, and 
                 the time to maturity is $1$ year. Its barrier appears in 
                 Figure~\ref{fig:partial_barrier}. The left exhibit shows the 
                 convergence of our algorithm. The true price is 11.6055 (dotted 
                 line). The right exhibit plots the speedups of our algorithm over 
                 the standard algorithm.}
        \label{fig:partial_barrier_option}
    \end{figure}

    Figure~\ref{fig:partial_barrier} illustrates the barrier of a partial-barrier 
    option with a one-year maturity. This barrier starts at the half year and lasts 
    for half a year with a level of 100. The true price is 11.6055 by the formula in 
    \cite{Heynen1994}.  
    %The Brownian-bridge approach with 10 million 
    %paths gives a mean price of 11.608 and a 
    %standard error of 0.006.
    Figure~\ref{fig:partial_barrier_option} depicts the convergence and speedups.
    %The initial stock price is $105$, the 
    %strike price is $110$, the annual growth 
    %rate is $10\%$, the volatility is $25\%$, 
    %the time to maturity is $1$ year. 
    Our algorithm converges very fast, and the pricing errors are at most 0.0004. 
    The speedups over the standard algorithm range from 7.05 to 25.86 for 
    $402 \leq n \leq 2802$ (in increments of 4).