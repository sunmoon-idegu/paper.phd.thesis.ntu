% !TeX root = ../main.tex

\chapter{Convergence Rates}\label{chap:convergence}
    
    This chapter provides the mathematical proof and the numerical results for the 
    linear convergence rate of the implied barrier from the core algorithm. 
    Section~\ref{sec:convergence-pd} proves that the default probability as 
    calculated on the $n$-step BTT converges to the continuous-time limit at a rate 
    of $O(n^{-1})$. Section~\ref{sec:convergence-ib} proves that, in the 
    constant-barrier case, the implied barrier level converges to the true barrier 
    level at a rate of $O(n^{-1})$. Section~\ref{sec:convergence-numerical} presents 
    the corresponding numerical results. We also show the numerical results of the 
    step-barrier case, which suggest a linear convergence rate as well. 
    
\section{Linear Convergence of the Default Probability}\label{sec:convergence-pd}

    For logarithmic stock price $x$ (which may depend on $n$) and time $t$, let 
    $C^\text{BS}(x,t)$ denote the DODC's expected payoff under the BS model and 
    $C^\text{BTT}_n(x,t)$ the DODC's expected payoff by the $n$-step BTT\@. Barrier 
    is ignored in the notations for brevity. The convergence rate of the DODC's 
    expected payoff by the CRR tree when the barrier passes through a terminal node 
    is covered by the next result, implicit in Lin and Palmer's proof of their 
    Theorem 1.2 (1)(b) \cite{Lin2013}.

    \begin{theorem}[\cite{Lin2013}]
        The expected payoff $C^{\textup{CRR}}_n$ of a DODC in the $n$-step CRR tree with 
        initial stock price $S_n$, strike price and barrier level equal to $H_n$, which  
        is a node on the grid, and maturity $T_n$ satisfies 
        $C^{\textup{CRR}}_n = C^{\textup{BS}} + O(n^{-1})$, where $C^{\textup{BS}}$ is 
        the expected payoff of the same DODC under the BS model, provided that $S_n/H_n$ 
        is bounded and $T_n\to T > 0$.
        \label{theorem:ddoc}
    \end{theorem}
    
    Let $P_\text{A}$, $P_\text{B}$, and $P_\text{C}$ be the survival probabilities on 
    the $n$-step BTT at time $T$ as seen from the root node R's three successor nodes 
    A, B, and C, respectively (recall Figure~\ref{fig:bino_tri_lattice_with_barrier}). 
    By equations~(\ref{eq:pu})--(\ref{eq:pd}),
    \begin{align}
        C^{\text{BTT}}_n\left(s_\text{R},0\right)\quad &=\quad P_uP_\text{A} + P_mP_\text{B} + P_dP_\text{C} \label{eq:sur_prob_T_a} \\[10pt] 
        %&=\quad \frac{\left(\beta-\sigma\sqrt{\Delta t}\right)^2}{8\sigma^2\Delta t}B_\text{A}\ +\ \frac{3\sigma^2\Delta t - \beta^2}{4\sigma^2\Delta t}B_\text{B}\ +\ \frac{\left(\beta+\sigma\sqrt{\Delta t}\right)^2}{8\sigma^2\Delta t}B_\text{C} \label{eq:sur_prob_T} \\[12pt]
        &=\quad \frac{P_\text{A}-2P_\text{B}+P_\text{C}}{8\sigma^2\Delta t}\,\beta^2 +\frac{-P_\text{A}+P_\text{C}}{4\sigma\sqrt{\Delta t}}\,\beta +\frac{P_\text{A}+6P_\text{B}+P_\text{C}}{8} \label{eq:sur_prob_T_b}
    \end{align}
    after simplification. Note that $P_\text{A}$, $P_\text{B}$, and $P_\text{C}$ are 
    independent of $\beta$ because (1) the transition probabilities of the CRR tree 
    are and (2) the number of grid points between nodes A, B, C, and the barrier, respectively,
    remain constant.
    Recall that $s_z$ stands for the logarithmic stock price of node $z$. Define
    \begin{align*}
        C^\text{BS}_x    \quad&\equiv\quad \frac{\partial   C^\text{BS}(s_\text{R},0)}{\partial x}, \\[3pt]
        C^\text{BS}_t    \quad&\equiv\quad \frac{\partial   C^\text{BS}(s_\text{R},0)}{\partial t}, \\[3pt]
        C^\text{BS}_{xx} \quad&\equiv\quad \frac{\partial^2 C^\text{BS}(s_\text{R},0)}{\partial x^2},
    \end{align*}
    and note that 
    \begin{align*}
        s_\text{A} \quad&=\quad s_\text{R} + \alpha + \theta,  \\[3pt]
        s_\text{B} \quad&=\quad s_\text{R} +  \beta + \theta,  \\[3pt]
        s_\text{C} \quad&=\quad s_\text{R} + \gamma + \theta
    \end{align*}
    by equations~(\ref{eq:alpha})-(\ref{eq:gamma}). Thus 
    \begin{align*}
        P_\text{A}\quad &=\quad C^\text{BS}\left(s_\text{A}, \Delta t\right)\ +\ O(\Delta t) \\
        &=\quad C^\text{BS}\left(s_\text{R}, 0\right)\ +\ C^\text{BS}_x\left(s_\text{A}-s_\text{R}\right)\ +\ C^\text{BS}_t\Delta t\ +\ \frac{C^\text{BS}_{xx}\left(s_\text{A}-s_\text{R}\right)^2}{2}\ +\ O(\Delta t) \\
        &=\quad C^\text{BS}\left(s_\text{R}, 0\right)\ +\ C^\text{BS}_x(\alpha+\theta)\ +\ C^\text{BS}_t\,\Delta t\ +\ \frac{C^\text{BS}_{xx}\left(\alpha+\theta\right)^2}{2}\ +\ O(\Delta t) \numberthis \label{eq:BA}
    \end{align*}
    by Theorem~\ref{theorem:ddoc} (recall that the survival probability equals the 
    expected payoff of the DODC) and the Taylor expansion around 
    $\left(s_\text{R}, 0\right)$. Similarly,
    \begin{align*}
        P_\text{B}\quad &=\quad C^\text{BS}\left(s_\text{R}, 0\right)\ +\ C^\text{BS}_x(\beta+\theta)\ +\  C^\text{BS}_t\,\Delta t\ +\ \frac{C^\text{BS}_{xx}\left(\beta+\theta\right)^2}{2}\ +\ O(\Delta t), \numberthis \label{eq:BB} \\
        P_\text{C}\quad &=\quad C^\text{BS}\left(s_\text{R}, 0\right)\ +\ C^\text{BS}_x(\gamma+\theta)\ +\ C^\text{BS}_t\,\Delta t\ +\ \frac{C^\text{BS}_{xx}\left(\gamma+\theta\right)^2}{2}\ +\ O(\Delta t). \numberthis \label{eq:BC}
    \end{align*}
    Finally, equation~(\ref{eq:sur_prob_T_a}) with the help of 
    equations~(\ref{eq:mean})--(\ref{eq:probs}) and (\ref{eq:BA})--(\ref{eq:BC}) reduces to
    \begin{align*}
        C^{\text{BTT}}_n\left(s_\text{R},0\right)\quad &=\quad C^{\text{BS}}(s_\text{R},0) + \theta C^{\text{BS}}_x + C^{\text{BS}}_t \Delta t + \frac{1}{2}\,C^{\text{BS}}_{xx}\left(\theta^2 + \sigma^2\Delta t\right) + O(\Delta t)\\
        &=\quad C^{\text{BS}}(s_\text{R},0) + O(\Delta t)
    \end{align*}
    as $\theta = O(\Delta t)$. We have proved that 
    $C^{\text{BTT}}_n(s_\text{R},0) - C^\text{BS}(s_\text{R},0) = O(n^{-1})$ 
    as $\Delta t = O(n^{-1})$. As default probability equals one minus the DODC's 
    expected payoff, we have the first main theorem below.
    \begin{theorem}
        \normalfont \textit{The default probability as calculated on the $n$-step BTT 
        converges to the continuous-time limit at a rate of} $O(n^{-1})$.
        \label{theorem:ddoc_BTT}
    \end{theorem}

\section{Linear Convergence of the Implied Barrier}\label{sec:convergence-ib}
    
    The following simple lemma will be useful later.
    \begin{lemma}\label{lemma:identical_BTT}
        Label a BTT's terminal nodes at $T$ from bottom to top as $\text{N}_0$, 
        $\text{N}_1$, $\ldots$, $\text{N}_{n+1}$. Consider two BTTs: one with 
        $\beta = -\sigma\sqrt{\Delta t}$ and the barrier level is aligned with the 
        terminal node $\text{N}_k$; and another with $\beta = \sigma\sqrt{\Delta t}$ 
        and the barrier level is aligned with $\text{N}_{k-1}$. The probabilities of 
        hitting the barrier are identical.
    \end{lemma}
    \begin{proof}
        With $\beta = -\sigma\sqrt{\Delta t}$, the transition probabilities are 0.5, 0.5, 
        and 0; with $\beta = \sigma\sqrt{\Delta t}$, the transition probabilities are 0, 
        0.5, and 0.5. So seen from the root node, the two BTTs are identical when their 
        barrier levels are as specified (see Figure~\ref{fig:proof_lemma}). 
    \end{proof}
    
    \begin{figure}[!t]
        \centering
        \resizebox{0.45\textwidth}{!}{
        \begin{tikzpicture}[myset/.list={(8,1),(7,2),(9,2),(11,2),
              (6,3),(8,3),(10,3),(12,3),
              (5,4),(7,4),(9,4),(11,4),(13,4),
              (4,5),(6,5),(8,5),(10,5),(12,5),(14,5),
              (3,6),(5,6),(7,6),(9,6),(11,6),(15,6)}, fillset/.list={(13,6)},
              fontset/.list={(17,1),(17,2),(17,3),(17,4),(17,5),(17,6)},
              left extended/.style={shorten <=-#1},
              left extended/.default=1cm,
              right extended/.style={shorten >=-#1},
              right extended/.default=1cm]
          \matrix (tree) [%
            nodes in empty cells,
            matrix of nodes,
            nodes={anchor=center,minimum size=0.75cm},
            column sep=1.5cm,
            row sep=0.5cm,
            ampersand replacement=\&
          ]
          {
                \&   \& \& \& \&   \\
                \&   \& \& \& \&   \\
                \&   \& \& \& \&   \\
                \&   \& \& \& \&   \\
                \&   \& \& \& \&   \\
                \&   \& \& \& \&   \\
                \&   \& \& \& \&   \\
                \&   \& \& \& \&   \\
                \&   \& \& \& \&   \\
                \&   \& \& \& \&   \\
                \&   \& \& \& \&   \\
                \&   \& \& \& \&   \\
                \&   \& \& \& \&   \\
                \&   \& \& \& \&   \\
                \&   \& \& \& \&   \\
              $|$ \& $|$ \& $|$ \& $|$ \& $|$ \& $|$ \\
          };
          \begin{pgfonlayer}{background}
              \draw[line width=1mm, left extended] (tree-13-1) -- (tree-13-6);
              \foreach \y in {1,...,6}{
                  \draw[dashed, left extended, right extended] (tree-1-\y) -- (tree-15-\y);
              }
              \foreach \x in {1,2,3,4,6,8,10,11,12,13,14,15}{
                  \draw[dashed, left extended, right extended] (tree-\x-2) -- (tree-\x-6);
              }
              \foreach \x in {5,7,9}{
                  \draw[dashed, left extended=1.06cm, right extended] (tree-\x-2) -- (tree-\x-6);
              }
              \draw[fill=black,thick,opacity=0.2] (tree-8-1.center) -- (tree-3-6.center) -- (tree-13-6.center) -- cycle ;
          \end{pgfonlayer}
          \node[shift={(1cm,0.4cm)},font=\Large] at (tree-13-6) {$\text{N}_1$};
          \node[below=0.5cm,font=\LARGE] at (tree-16-1) {$0$};
          \node[below=0.5cm,font=\LARGE] at (tree-16-2) {$\Delta t$};
          \node[below=0.5cm,font=\LARGE] at (tree-16-3) {$2\Delta t$};
          \node[below=0.5cm,font=\LARGE] at (tree-16-4) {$3\Delta t$};
          \node[below=0.5cm,font=\LARGE] at (tree-16-5) {$4\Delta t$};
          \node[below=0.5cm,font=\LARGE] at (tree-16-6) {$5\Delta t$};
          \node[below=1.2cm,font=\LARGE] at (tree-16-6) {$(=T)$};
          \draw[-{Latex[length=2mm]}] (tree-8-1) -- (tree-7-2) node [midway,above=3mm,font=\LARGE] {$1/2$}; 
          \draw[-{Latex[length=2mm]}] (tree-8-1) -- (tree-9-2) node [midway,below=1mm,font=\LARGE] {$1/2$}; 
          \draw[dashed,-{Latex[length=2mm]}] (tree-8-1) -- (tree-11-2) node [midway,below,font=\LARGE] {$0$};
          \draw[-{Latex[length=2mm]}] (tree-7-2) -- (tree-6-3); 
          \draw[-{Latex[length=2mm]}] (tree-7-2) -- (tree-8-3); 
          \draw[-{Latex[length=2mm]}] (tree-9-2) -- (tree-8-3); 
          \draw[-{Latex[length=2mm]}] (tree-9-2) -- (tree-10-3); 
          \draw[-{Latex[length=2mm]}] (tree-11-2) -- (tree-10-3); 
          \draw[-{Latex[length=2mm]}] (tree-11-2) -- (tree-12-3); 
          \draw[-{Latex[length=2mm]}] (tree-6-3) -- (tree-5-4); 
          \draw[-{Latex[length=2mm]}] (tree-6-3) -- (tree-7-4); 
          \draw[-{Latex[length=2mm]}] (tree-8-3) -- (tree-7-4); 
          \draw[-{Latex[length=2mm]}] (tree-8-3) -- (tree-9-4); 
          \draw[-{Latex[length=2mm]}] (tree-10-3) -- (tree-9-4); 
          \draw[-{Latex[length=2mm]}] (tree-10-3) -- (tree-11-4); 
          \draw[-{Latex[length=2mm]}] (tree-12-3) -- (tree-11-4); 
          \draw[-{Latex[length=2mm]}] (tree-12-3) -- (tree-13-4); 
          \draw[-{Latex[length=2mm]}] (tree-5-4) -- (tree-4-5); 
          \draw[-{Latex[length=2mm]}] (tree-5-4) -- (tree-6-5); 
          \draw[-{Latex[length=2mm]}] (tree-7-4) -- (tree-6-5); 
          \draw[-{Latex[length=2mm]}] (tree-7-4) -- (tree-8-5); 
          \draw[-{Latex[length=2mm]}] (tree-9-4) -- (tree-8-5); 
          \draw[-{Latex[length=2mm]}] (tree-9-4) -- (tree-10-5); 
          \draw[-{Latex[length=2mm]}] (tree-11-4) -- (tree-10-5); 
          \draw[-{Latex[length=2mm]}] (tree-11-4) -- (tree-12-5); 
          \draw[-{Latex[length=2mm]}] (tree-13-4) -- (tree-12-5); 
          \draw[-{Latex[length=2mm]}] (tree-13-4) -- (tree-14-5); 
          \draw[-{Latex[length=2mm]}] (tree-4-5) -- (tree-3-6); 
          \draw[-{Latex[length=2mm]}] (tree-4-5) -- (tree-5-6); 
          \draw[-{Latex[length=2mm]}] (tree-6-5) -- (tree-5-6); 
          \draw[-{Latex[length=2mm]}] (tree-6-5) -- (tree-7-6); 
          \draw[-{Latex[length=2mm]}] (tree-8-5) -- (tree-7-6); 
          \draw[-{Latex[length=2mm]}] (tree-8-5) -- (tree-9-6); 
          \draw[-{Latex[length=2mm]}] (tree-10-5) -- (tree-9-6); 
          \draw[-{Latex[length=2mm]}] (tree-10-5) -- (tree-11-6); 
          \draw[-{Latex[length=2mm]}] (tree-12-5) -- (tree-11-6); 
          \draw[-{Latex[length=2mm]}] (tree-12-5) -- (tree-13-6); 
          \draw[-{Latex[length=2mm]}] (tree-14-5) -- (tree-13-6); 
          \draw[-{Latex[length=2mm]}] (tree-14-5) -- (tree-15-6); 
          \draw (tree-16-1.mid) -- (tree-16-6.mid);
        \end{tikzpicture}}
        \hspace{0.05\textwidth}
        \resizebox{0.45\textwidth}{!}{
        \begin{tikzpicture}[myset/.list={(8,1),(5,2),(7,2),(9,2),
              (4,3),(6,3),(8,3),(10,3),
              (3,4),(5,4),(7,4),(9,4),(11,4),
              (2,5),(4,5),(6,5),(8,5),(10,5),(12,5),
              (1,6),(3,6),(5,6),(7,6),(9,6),(11,6)}, fillset/.list={(13,6)},
              fontset/.list={(15,1),(15,2),(15,3),(15,4),(15,5),(15,6)},
              left extended/.style={shorten <=-#1},
              left extended/.default=1cm,
              right extended/.style={shorten >=-#1},
              right extended/.default=1cm]
          \matrix (tree) [%
            nodes in empty cells,
            matrix of nodes,
            nodes={anchor=center,minimum size=0.75cm},
            column sep=1.5cm,
            row sep=0.5cm,
            ampersand replacement=\&
          ]
          {
                \&   \& \& \& \&   \\
                \&   \& \& \& \&   \\
                \&   \& \& \& \&   \\
                \&   \& \& \& \&   \\
                \&   \& \& \& \&   \\
                \&   \& \& \& \&   \\
                \&   \& \& \& \&   \\
                \&   \& \& \& \&   \\
                \&   \& \& \& \&   \\
                \&   \& \& \& \&   \\
                \&   \& \& \& \&   \\
                \&   \& \& \& \&   \\
                \&   \& \& \& \&   \\
                \&   \& \& \& \&   \\
                \&   \& \& \& \&   \\
              $|$ \& $|$ \& $|$ \& $|$ \& $|$ \& $|$ \\
          };
          \begin{pgfonlayer}{background}
              \draw[line width=1mm, left extended] (tree-13-1) -- (tree-13-6);
              \foreach \y in {1,...,6}{
                  \draw[dashed, left extended, right extended] (tree-1-\y) -- (tree-15-\y);
              }
              \foreach \x in {1,2,3,4,6,8,10,11,12,13,14,15}{
                  \draw[dashed, left extended, right extended] (tree-\x-2) -- (tree-\x-6);
              }
              \foreach \x in {5,7,9}{
                  \draw[dashed, left extended=1.06cm, right extended] (tree-\x-2) -- (tree-\x-6);
              }
              \draw[fill=black,thick,opacity=0.2] (tree-8-1.center) -- (tree-3-6.center) -- (tree-13-6.center) -- cycle;
          \end{pgfonlayer}
          \node[shift={(1cm,0.4cm)},font=\Large] at (tree-13-6) {$\text{N}_0$};
          \node[below=0.5cm,font=\LARGE] at (tree-16-1) {$0$};
          \node[below=0.5cm,font=\LARGE] at (tree-16-2) {$\Delta t$};
          \node[below=0.5cm,font=\LARGE] at (tree-16-3) {$2\Delta t$};
          \node[below=0.5cm,font=\LARGE] at (tree-16-4) {$3\Delta t$};
          \node[below=0.5cm,font=\LARGE] at (tree-16-5) {$4\Delta t$};
          \node[below=0.5cm,font=\LARGE] at (tree-16-6) {$5\Delta t$};
          \node[below=1.2cm,font=\LARGE] at (tree-16-6) {$(=T)$};
          \draw[dashed,-{Latex[length=2mm]}] (tree-8-1) -- (tree-5-2) node [midway,above=3mm,font=\LARGE] {$0$}; 
          \draw[-{Latex[length=2mm]}] (tree-8-1) -- (tree-7-2) node [midway,below=1mm,font=\LARGE] {$1/2$}; 
          \draw[-{Latex[length=2mm]}] (tree-8-1) -- (tree-9-2) node [midway,below=2mm,font=\LARGE] {$1/2$};
          \draw[-{Latex[length=2mm]}] (tree-5-2) -- (tree-4-3); 
          \draw[-{Latex[length=2mm]}] (tree-5-2) -- (tree-6-3); 
          \draw[-{Latex[length=2mm]}] (tree-7-2) -- (tree-6-3); 
          \draw[-{Latex[length=2mm]}] (tree-7-2) -- (tree-8-3); 
          \draw[-{Latex[length=2mm]}] (tree-9-2) -- (tree-8-3); 
          \draw[-{Latex[length=2mm]}] (tree-9-2) -- (tree-10-3); 
          \draw[-{Latex[length=2mm]}] (tree-4-3) -- (tree-3-4); 
          \draw[-{Latex[length=2mm]}] (tree-4-3) -- (tree-5-4); 
          \draw[-{Latex[length=2mm]}] (tree-6-3) -- (tree-5-4); 
          \draw[-{Latex[length=2mm]}] (tree-6-3) -- (tree-7-4); 
          \draw[-{Latex[length=2mm]}] (tree-8-3) -- (tree-7-4); 
          \draw[-{Latex[length=2mm]}] (tree-8-3) -- (tree-9-4); 
          \draw[-{Latex[length=2mm]}] (tree-10-3) -- (tree-9-4); 
          \draw[-{Latex[length=2mm]}] (tree-10-3) -- (tree-11-4); 
          \draw[-{Latex[length=2mm]}] (tree-3-4) -- (tree-2-5); 
          \draw[-{Latex[length=2mm]}] (tree-3-4) -- (tree-4-5); 
          \draw[-{Latex[length=2mm]}] (tree-5-4) -- (tree-4-5); 
          \draw[-{Latex[length=2mm]}] (tree-5-4) -- (tree-6-5); 
          \draw[-{Latex[length=2mm]}] (tree-7-4) -- (tree-6-5); 
          \draw[-{Latex[length=2mm]}] (tree-7-4) -- (tree-8-5); 
          \draw[-{Latex[length=2mm]}] (tree-9-4) -- (tree-8-5); 
          \draw[-{Latex[length=2mm]}] (tree-9-4) -- (tree-10-5); 
          \draw[-{Latex[length=2mm]}] (tree-11-4) -- (tree-10-5); 
          \draw[-{Latex[length=2mm]}] (tree-11-4) -- (tree-12-5); 
          \draw[-{Latex[length=2mm]}] (tree-2-5) -- (tree-1-6); 
          \draw[-{Latex[length=2mm]}] (tree-2-5) -- (tree-3-6); 
          \draw[-{Latex[length=2mm]}] (tree-4-5) -- (tree-3-6); 
          \draw[-{Latex[length=2mm]}] (tree-4-5) -- (tree-5-6); 
          \draw[-{Latex[length=2mm]}] (tree-6-5) -- (tree-5-6); 
          \draw[-{Latex[length=2mm]}] (tree-6-5) -- (tree-7-6); 
          \draw[-{Latex[length=2mm]}] (tree-8-5) -- (tree-7-6); 
          \draw[-{Latex[length=2mm]}] (tree-8-5) -- (tree-9-6); 
          \draw[-{Latex[length=2mm]}] (tree-10-5) -- (tree-9-6); 
          \draw[-{Latex[length=2mm]}] (tree-10-5) -- (tree-11-6); 
          \draw[-{Latex[length=2mm]}] (tree-12-5) -- (tree-11-6); 
          \draw[-{Latex[length=2mm]}] (tree-12-5) -- (tree-13-6); 
          \draw (tree-16-1.mid) -- (tree-16-6.mid); 
        \end{tikzpicture}}                                                                   
        \caption{Two identical BTTs: $\beta = -\sigma\sqrt{\Delta t}$ (left) and 
                 $\beta = \sigma\sqrt{\Delta t}$ (right). The thick black lines
                 indicate the barriers.}
        \label{fig:proof_lemma}
    \end{figure}
    
    %\begin{figure}[!t]
    %    \centering
    %    \includegraphics[width=0.47\textwidth]{./figures/binary1.pdf}
    %    \includegraphics[width=0.47\textwidth]{./figures/binary2.pdf}
    %    \caption{The default probability as a concave quadratic function of $\beta$.
    %             The function value starts at $p_i$ and ends at $p_{i+1}$ by Lemma~\ref{lemma:identical_BTT}.
    %             The function's maximum may occur before (the left panel) or after
    %             (the right panel) $\sigma\sqrt{\Delta t}$.
    %             }
    %    \label{fig:lemma1}
    %\end{figure}

    The following identities are clear from equations~(\ref{eq:alpha})--(\ref{eq:gamma}):
    \begin{align}
        \alpha - \gamma               \quad &= \quad 4\sigma\sqrt{\Delta t}, \numberthis \label{eq:alpha-gamma} \\[3pt]
        \alpha - 2\beta + \gamma      \quad &= \quad 0,                      \numberthis \label{eq:alphal+gamma-2beta} \\[3pt]
        \alpha^2 - 2\beta^2 + \gamma^2\quad &= \quad 8\sigma^2\Delta t.      \numberthis \label{eq:alpha2+gamma2-2beta2}
    \end{align}
    Equation~(\ref{eq:sur_prob_T_b}) with the help of 
    equations~(\ref{eq:alpha})--(\ref{eq:gamma}) and 
    (\ref{eq:BA})--(\ref{eq:alpha2+gamma2-2beta2}) becomes
    \begin{align*}
        &C^{\text{BTT}}_n(s_\text{R},0) \\[10pt]
        =\quad &\frac{C^\text{BS}_{xx}\sigma^2\Delta t+O(\Delta t)}{2\sigma^2\Delta t}\,\beta^2 -\frac{\sigma\sqrt{\Delta t}\left(C^\text{BS}_x+C^\text{BS}_{xx}\hat\theta\right)+O(\Delta t)}{\sigma\sqrt{\Delta t}}\,\beta +\frac{P_\text{A}+6P_\text{B}+P_\text{C}}{8}. \numberthis \label{eq:CDO_new}
    \end{align*}
    \begin{lemma}
        The expected payoff of a DODC by the BTT is a quadratic function of $\beta$. 
        Furthermore, the coefficient of $\beta^2$ and $\beta$ are $O(1)$, and the 
        $\beta$ term dominates the $\beta^2$ term in magnitude.
        \label{lemma:beta_coefficients}
    \end{lemma}
    \begin{proof}
        The expected payoff is quadratic in $\beta$ by equation~(\ref{eq:sur_prob_T_b}). 
        The coefficients of $\beta^2$ and $\beta$ equal $C^\text{BS}_{xx}/2 + O(1)$ and 
        $-C^\text{BS}_x - C^\text{BS}_{xx}\hat\theta + O(\sqrt{\Delta t})$, respectively, 
        by equation~(\ref{eq:CDO_new}). Both coefficients are $O(1)$ because 
        $\hat\theta = O(\sqrt{\Delta t})$ and $C^\text{BS}_x$ and $C^\text{BS}_{xx}$ are 
        bounded. Furthermore, the coefficient of $\beta$ approaches $-C^\text{BS}_x$, 
        which is negative and bounded away from zero. The $\beta$ term dominates the 
        $\beta^2$ term in magnitude because $\beta^2 \ll \beta$ for
        $\beta = O(\sqrt{\Delta t})$.
    \end{proof}
        
    Let $Q_k$ be the DODC's expected payoff by the $n$-step BTT with 
    $\beta = -\sigma\sqrt{\Delta t}$ when terminal node $\text{N}_k$ is the barrier 
    level (recall Figure~\ref{fig:algor_part_1}). Note that $Q_k = 1 - p_k$ and
    $Q_0 > Q_1 > \cdots$ until the expected payoff reaches zero, where it stays. We 
    are now in a position to prove the main result: The exact implied barrier level 
    differs from the true barrier level by $O(n^{-1})$.
    \begin{theorem}
        The exact implied barrier level by the $n$-step BTT differs from the true 
        barrier level by $O(n^{-1})$.
        \label{theorem:constant_barrier_convergence}
    \end{theorem}
    \begin{proof}
        Let $p$ be the default probability. Denote the logarithmic value of the exact 
        implied barrier level by $b^{\text{imp}}_n$ and the true logarithmic barrier level 
        by $b^{\text{true}}$, respectively. Note that $b^{\text{true}}$ is a constant and 
        below the initial stock price by our convention. Although the goal is 
        $e^{b^{\text{imp}}_n} - e^{b^{\text{true}}} = O(n^{-1})$, it suffices to prove 
        $b^{\text{imp}}_n - b^{\text{true}} = O(n^{-1})$ because, then, 
        \begin{equation*}
            e^{b^{\text{imp}}_n} - e^{b^{\text{true}}}\ =\ e^{b^{\text{imp}}_n}\left(1 - e^{b^{\text{true}}-b^{\text{imp}}_n}\right)\ \approx\ e^{b^{\text{imp}}_n}\left(b^{\text{imp}}_n-b^{\text{true}}\right),
        \end{equation*}
        which converges to zero at a speed of $n^{-1}$.

        The DODC's expected payoff $q^{\text{imp}}_n$ induced by $b^{\text{imp}}_n$ on the 
        $n$-step BTT equals $1-p$ as $b^{\text{imp}}_n$ is exact. Assume the algorithm for 
        the implied barrier ends with $\beta = \beta'$. Let $q^{\text{true}}$ denote the 
        DODC's expected payoff induced by $b^{\text{true}}$ on the $n$-step BTT which
        ends with $\beta = \beta''$. 
        %The scenario of $q = q_{\text{true}}$ is trivial 
        %because they give rise to the same BTT if the initial stock price is the 
        %same, which yields an exact implied barrier level equal to the true barrier 
        %level by definition. 
        Without loss of generality, assume $q^{\text{imp}}_n \geq q^{\text{true}}$; the 
        argument is symmetric if $q^{\text{imp}}_n \leq q^{\text{true}}$. Recall that the 
        default probability equals one minus the DODC's expected payoff. 
        Theorem~\ref{theorem:ddoc_BTT} says
        \begin{equation}
            q^{\text{imp}}_n - q^{\text{true}}\ =\ O\left(n^{-1}\right).
            \label{eq:qq_imp_true}
        \end{equation}
        Let
        \begin{alignat*}{2}
            Q_{i+1}\quad &\leq \quad q^{\text{imp}}_n \quad &&< \quad Q_i,\\[3pt]
            Q_{j+1}\quad &\leq \quad q^{\text{true}}  \quad &&< \quad Q_j. 
        \end{alignat*}
        Because $q^{\text{imp}}_n \geq q^{\text{true}}$, there are three cases to 
        consider: (1) $i = j$, (2) $i+1 = j$, and (3) $i+1 < j$.
        
        For case (1), 
        \begin{equation*}
            b^{\text{true}} - b^{\text{imp}}_n\ =\ \beta'' - \beta'\ 
        \end{equation*}
        (see Figure~\ref{fig:case1}). Therefore
        \begin{equation*}
            \left\vert\,b^{\text{imp}}_n - b^{\text{true}}\,\right\vert\ =\ O\left(q^{\text{imp}}_n - q^{\text{true}}\right)\ =\ O\left(n^{-1}\right),
        \end{equation*}
        as desired, by equation~(\ref{eq:qq_imp_true}) and Lemma~\ref{lemma:beta_coefficients}.
        For case (2), 
        \begin{equation*}
            Q_{i+2}\ \leq\ q^\text{true}\ <\ Q_{i+1}\ \leq\ q^\text{imp}_n\ <\ Q_{i};
            \label{eq:case2}
        \end{equation*}
        hence it is necessary that 
        \begin{align*}
            q^{\text{imp}}_n - Q_{i+1}\ =\ O\left(n^{-1}\right), \\[3pt]
            Q_{i+1} - q^{\text{true}} \ =\ O\left(n^{-1}\right)
        \end{align*}
        by equation~(\ref{eq:qq_imp_true}) (see Figure~\ref{fig:case2}).
        \begin{figure}[!t]
            \centering
            \includegraphics[width=\textwidth]{./figures/case1.pdf}
            \caption{Relations among the six quantities $Q_i$, $Q_{i+1}$, $q^{\text{imp}}_n$,
                     $q^{\text{true}}$, $\beta'$ and $\beta''$ for case (1).}
            \label{fig:case1}
        \end{figure}
        Apply the same argument as in case (1) to obtain, separately,
        \begin{align*}
            \left\vert\,b^{\text{imp}}_n - \text{N}_{i+1}\,\right\vert\ &=\ O\left(q^{\text{imp}}_n - Q_{i+1}\right)\ =\ O\left(n^{-1}\right), \\[3pt]
            \left\vert\,b^{\text{true}}  - \text{N}_{i+1}\,\right\vert\ &=\ O\left(q^{\text{true}}  - Q_{i+1}\right)\  =\ O\left(n^{-1}\right).
        \end{align*}
        Finally, 
        \begin{equation*}
            \left\vert\,b^{\text{imp}}_n - b^{\text{true}}\,\right\vert\ \leq\ \left\vert\,b^{\text{imp}}_n - \text{N}_{i+1}\,\right\vert + \left\vert\,\text{N}_{i+1} - b^{\text{true}}\,\right\vert\ =\ O\left(n^{-1}\right),
        \end{equation*}
        as desired. For case (3),
        \begin{equation*}
            q^\text{true}\ <\ Q_{j}\ <\ Q_{i+1} \leq\ q^\text{imp}_n.
        \end{equation*}
        The above inequalities imply 
        $q^{\text{imp}}_n - q^{\text{true}} > Q_{i+1} - Q_{j}$, which is of order
        $n^{-0.5}$ for a constant barrier level, contradicting equation~(\ref{eq:qq_imp_true}).
        
        \begin{figure}[!t]
            \centering
            \includegraphics[width=\textwidth]{./figures/case2.pdf}
            \caption{Relations among the five quantities $Q_i$, $Q_{i+1}$, $Q_{i+2}$,
                     $q^{\text{imp}}_n$, and $q^{\text{true}}$ for case (2).}
            \label{fig:case2}
        \end{figure}
        
    \end{proof}

\section{Numerical Results}\label{sec:convergence-numerical}

    The first experiment employs the $n$-step BTT to calculate the expected payoff of 
    a DODC --- equivalently, one minus the default probability --- for 
    $n = 3, 5, \ldots, 101$. The initial stock price is one dollar, the barrier level 
    is 0.9, the return rate is 10\%, the volatility is 25\%, and the time to maturity 
    is one year. We compare our numbers with 0.407, which is obtained by the
    \citetalias{ReinerRubinstein1991} formula. The left panel of 
    Figure~\ref{fig:price_barrier_convergence} plots the DODC's expected payoffs; the 
    right panel plots the corresponding absolute errors. The convergence rate is 
    clearly linear, consistent with Theorem~\ref{theorem:ddoc_BTT}.
    
    \begin{figure*}[!t]
    \centering
        \includegraphics[width = 0.485\textwidth]{figures/experiment_expected_payoff_9.pdf}
        \includegraphics[width = 0.475\textwidth]{figures/experiment_expected_payoff_error_9.pdf}
        \caption{The expected payoff of a DODC and its absolute error when the barrier level is 0.9.}
        \label{fig:price_barrier_convergence}
    \end{figure*}
    
    The second experiment calculates the implied barrier by the $n$-step BTT\@. The 
    parameters are the same as above except that the expected payoff 0.407 is given 
    in place of the barrier level. The left panel of 
    Figure~\ref{fig:barrier_convergence} plots the implied barriers; the right panel 
    plots the absolute errors of the implied barrier levels from 0.9. The convergence 
    rate is again linear, consistent with 
    Theorem~\ref{theorem:constant_barrier_convergence}. 

    \begin{figure*}[!t]
    \centering
        \includegraphics[width = 0.485\textwidth]{figures/experiment_implied_barrier_9.pdf}
        \includegraphics[width = 0.475\textwidth]{figures/experiment_implied_barrier_error_9.pdf}
        \caption{The implied barrier level and its absolute error from the true barrier level 0.9.}
        \label{fig:barrier_convergence}
    \end{figure*}
    
    %The BTT-based algorithms of Lu and Lyuu work for the more complex step barriers. 
    It is interesting to see how fast the expected payoff and the implied step 
    barrier converge even in the absence of proofs. We focus on the implied step 
    barrier as it is built upon the expected payoff and therefore cannot better the 
    latter's convergence rate. By concatenating twelve equal-length (thus each one 
    month long) contiguous barriers at the same level 0.9, we effectively create a 
    constant-barrier case for which Reiner and Rubinstein’s closed-form formula again 
    applies. The algorithm, of course, treats the barrier as a step barrier not as 
    one single constant barrier. The parameters are otherwise the same as above. 
    (A similar setup was used in Figure~\ref{fig:step_barrier_convergence}.) The 
    left panel of Figure~\ref{fig:implied_step_barrier} shows the implied step 
    barriers obtained for $n = 36, 156, 396, 1212$\@. The implied step barriers 
    converge to the true barrier as $n$ increases. The right panel of 
    Figure~\ref{fig:implied_step_barrier} plots the maximum absolute errors of the 
    implied step barriers from 0.9 for $n = 36, 60, \ldots, 1212$. The maximum 
    absolute errors by treating the step barrier as one constant barrier are 
    superimposed for a quick comparison. Both exhibit a linear convergence rate. 
    The larger error in the step-barrier version compared with the constant-barrier 
    version can be explained by the fact that each of its twelve barrier levels 
    receives only one-twelfth as many time steps as the constant-barrier version.

    \begin{figure*}[!t]
    \centering
        \includegraphics[width = 0.485\textwidth]{figures/experiment_step_barrier_convergence_0.9.pdf}
        \includegraphics[width = 0.475\textwidth]{figures/experiment_step_barrier_absolute_error_0.9.pdf}
        \caption{The implied step barrier and its maximum absolute error when the true step barrier 
                 consists of twelve consecutive identical barrier levels at 0.9. 
                 The absolute error by treating the step barrier as one single constant barrier 
                 is also plotted for comparison (the right panel).}
        \label{fig:implied_step_barrier}
    \end{figure*}