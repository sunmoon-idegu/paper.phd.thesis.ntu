% !TeX root = ../main.tex

\chapter{Preliminaries}\label{chap:preliminaries}

    This chapter introduces the fundamental concepts relevant to our thesis.
    Section~\ref{sec:preliminaries-term} defines term sturcture of default 
    probabilites. Section~\ref{sec:preliminaries-rnp} describes risk-neutral option 
    pricing. Section~\ref{sec:preliminaries-bo} introduces barrier options.
    Section~\ref{sec:preliminaries-reflection} explains the reflection principle on 
    a binomial random walk. Section~\ref{sec:preliminaries-bt} discusses the CRR 
    tree. Section~\ref{sec:preliminaries-btt} elaborates on the binomial-trinomial 
    tree, which forms the computational structure of the thesis. 

\section{Term Structure of Default Probabilities}\label{sec:preliminaries-term}

    Define $\tau$ as a nonnegative random variable representing a company's default 
    time in the future. The term structure of default probabilities is the function 
    \begin{equation*}
        p(t\,|\,t_0) = \text{Pr}(\tau \leq t_0 + t\,|\,t_0),
    \end{equation*}
    the cumulative probability of defaulting at or before time $t$, observed at time 
    $t_0$ when the company is not in default (see~\cite{Lando2004} and 
    \cite{Bluhm2010}). We refer to $t_0+t$ as the monitored time point. For 
    simplicity, we will omit the term ``cumulative'' and call $p(t\,|\,t_0)$ the 
    default probability over the time interval $(\,t_0, t_0+t\,]$. 

\section{Risk-Neutral Pricing}\label{sec:preliminaries-rnp}
    
    Risk-neutral pricing is critical to pricing financial derivatives like options. 
    It says the price of a derivative equals its expected payoff discounted at the 
    risk-free interest rate (see~\cite{Hull2006}). In other words, given a payoff 
    $C_T$ at maturity $T$, its theoretical price $C_0$ now equals
    \begin{equation}
        C_0 = e^{-rT}E^Q\left[C_T\right],
        \label{eq:risk_neutral}
    \end{equation}
    where $E^Q[\,\cdot\,]$ is the expectation under the risk-neutral probability 
    measure $Q$.

    In the BS model, a stock price $S$ follows a geometric Brownian motion under the 
    real-world probability measure $P$:
    \begin{equation}
        dS\ =\ \mu Sdt + \sigma S\,dW^P,
        \label{eq:stock_process}
    \end{equation}
    where $\mu$ is the the growth rate of the stock price return, $\sigma$ is the 
    volatility, and $W^P$ is the Wiener process under the $P$ measure. By Ito's 
    lemma, process~(\ref{eq:stock_process}) becomes 
    \begin{equation}
        d\text{log}S\ =\ \left(\mu - \frac{\sigma^2}{2}\,\right)dt + \sigma\,dW^P
        \label{eq:log_stock_process}
    \end{equation}
    (see~\cite{Shreve2004}).

    Under the $Q$ measure, where all assets are expected to grow at the risk-free 
    interest rate $r$, process~(\ref{eq:log_stock_process}) becomes 
    \begin{equation*}
        d\text{log}S\ =\ \left(r - \frac{\sigma^2}{2}\,\right)dt + \sigma\,dW^Q,
    \end{equation*}
    where $W^Q$ is the Wiener process under the $Q$ measure.
    
\section{Barrier Option}\label{sec:preliminaries-bo}
    
    A barrier option is a vanilla option plus a barrier that triggers a specific 
    event. Its payoff depends on whether the underlying stock's price reaches a given 
    level at any time before maturity. Depending on the barrier type, the option can 
    be activated (the knock-in type) or terminated (the knock-out type) when the 
    barrier is touched. We focus on the latter case. The most basic example is the 
    down-and-out call (DOC). A DOC is terminated if the barrier is touched by the 
    underlying stock's price from above. The payoff of a typical DOC is therefore
    \begin{equation}
        C^{\text{DOC}}_T = \max\left(S_T - K, 0\right)\mathbbm{1}\left\llbracket S_t > B_t \quad \text{for all}\,t\in [\,0, T\,]\right\rrbracket,
        \label{eq:doc_payoff}
    \end{equation}
    where $S_t$ is the stock price at time $t$, $K$ is the strike price, $B_t$ is the 
    barrier at time $t$, and $\mathbbm{1}$ is the indicator function. The DODC 
    mentioned in Section~\ref{sec:introduction-background} has a similar but simpler 
    payoff:
    \begin{equation*}
        C^{\text{DODC}}_T = \mathbbm{1}\left\llbracket S_t > B_t \quad \text{for all}\,t\in [\,0, T\,]\right\rrbracket.
        \label{eq:dodc_payoff}
    \end{equation*}

\section{The Reflection Principle}\label{sec:preliminaries-reflection}
    
    A lattice consists of rectangular lines. The intersection of two lines defines a 
    lattice point. A binomial random walk on the lattice allows a move from position 
    $(i, j)$ to $(i+1, j+1)$ or $(i+1, j-1)$ per step. We now count the number of 
    walks that start from $(0, u)$ and end at $(n, v)$ while hitting the $x$-axis in 
    the process. The reflection principle says that number equals the number of 
    walks from $(0, -u)$ to $(n, v)$ when $n-u-v$ is even and $u > 0$. This is 
    because reflecting the part of the walk before it touches the $x$-axis for the 
    first time yields a walk from $(0, -u)$ in a one-to-one manner (see 
    Figure~\ref{fig:reflection_principle}). Because such walks must contain $v+u$ 
    more up moves than down moves in $n$ steps, the desired count totals
    \begin{equation}
        \frac{n!}{\left[\,\left(n+u+v\right)/2\,\right]!\left[\,\left(n-u-v\right)/2\,\right]!}\ =\ \binom{n}{\frac{n-u-v}{2}}
        \label{eq:reflection_principle}
    \end{equation}
    and zero when $n-u-v < 0$ or $n-u-v > n$ (see~\cite{Lyuu1998}). 
        
    \begin{figure}[!t]
        \centering
        \begin{tikzpicture}
            \draw[step=1cm,color=gray,dotted] (.75,.75) grid (12.25,5.25);
            \node[left=0.2cm] at (1,1){$(0,-u)$};
            \node[left=0.2cm] at (1,5){$(0,u)$};
            \node[right=0.2cm] at (12,4){$(n,v)$};
            \draw[-] (1,5) -- (2,4); 
            \draw[-] (2,4) -- (3,5); 
            \draw[-] (3,5) -- (4,4); 
            \draw[-] (4,4) -- (5,3); 
            \draw[-] (5,3) -- (6,4); 
            \draw[-] (6,4) -- (7,3); 
            \draw[-] (7,3) -- (8,4); 
            \draw[-] (8,4) -- (9,5); 
            \draw[-] (9,5) -- (10,4); 
            \draw[-] (10,4) -- (11,3); 
            \draw[-{Latex[length=2mm]}] (11,3) -- (12,4); 
            \draw[dashed] (1,1) -- (2,2); 
            \draw[dashed] (2,2) -- (3,1); 
            \draw[dashed] (3,1) -- (4,2); 
            \draw[dashed] (4,2) -- (5,3); 
            \draw[-, line width=0.25mm] (0.75,3) -- (12.25,3); 
        \end{tikzpicture}
        \caption{The reflection principle. The thick horizontal line denotes the $x$-axis.}
        \label{fig:reflection_principle}
    \end{figure}

\section{The CRR Tree}\label{sec:preliminaries-bt}

    Assume the logarithmic stock price follows process~(\ref{eq:log_stock_process}). 
    A binomial tree discretizes process~(\ref{eq:log_stock_process}) from $t$ to $T$ 
    by dividing it evenly into $n$ time steps; thus each time step has a duration of 
    $\Delta t \equiv (T-t)/n$. The logarithmic stock price $\text{log}S$ can either 
    go upward to $\text{log}S + \text{log}u$ with probability $p$ or downward to 
    $\text{log}S + \text{log}d$ with probability $1-p$ per step, where $u$, $d$, and 
    $p$ are chosen to match the mean and variance of 
    process~(\ref{eq:log_stock_process}) asymptotically. These parameters are chosen
    by the CRR to be $u \equiv e^{\sigma\sqrt{\Delta t}}$, 
    $d \equiv e^{-\sigma\sqrt{\Delta t}}$, and 
    $p \equiv (e^{\mu \Delta t} - d)/(u - d)$. They are attractive because the 
    property $ud = 1$ is convenient for our base case, where the barrier is a 
    constant barrier or a step function. Figure~\ref{fig:btree} illustrates a CRR 
    tree for $n = 2$. Note that the spacing between two vertically adjacent nodes is 
    $2\sigma\sqrt{\Delta t}\,$. For the risk-neutral measure needed for options 
    pricing, simply set $\mu = r$, the risk-free interest rate.

    The growth rate and the volatility in process~(\ref{eq:log_stock_process}) are 
    constants. In more general models, these parameters can be functions of time, 
    denoted by $\mu(t)$ and $\sigma(t)$ (see~\cite{LyuuZhang2023}).

    \begin{figure}[!t]
        \centering
        \begin{tikzpicture}
            \matrix (tree) [%
              nodes in empty cells,
              matrix of nodes,
              column sep=2cm,
              row sep=1.5cm,
              ampersand replacement=\&
            ]
            {
                             \&                             \& $\text{log}S$ + 2$\text{log}u$ \&[-1.5cm]\\
                             \& $\text{log}S$ + $\text{log}u$ \&                              \&[-1.5cm]\\
                $\text{log}S$ \&                             \& $\text{log}S$                 \&[-1.5cm]\\
                             \& $\text{log}S$ + $\text{log}d$ \&                              \&[-1.5cm]\\
                             \&                             \& $\text{log}S$ + 2$\text{log}d$ \&[-1.5cm]\\
                         $|$ \&           $|$               \&             $|$              \&[-1.5cm]\\
            };
            \draw[-{Latex[length=2mm]}] (tree-3-1) -- (tree-2-2) node[midway,above] {$p$};
            \draw[-{Latex[length=2mm]}] (tree-3-1) -- (tree-4-2) node[midway,above] {$1-p$};
            \draw[-{Latex[length=2mm]}] (tree-2-2) -- (tree-1-3) node[midway,above] {$p$};
            \draw[-{Latex[length=2mm]}] (tree-2-2) -- (tree-3-3) node[midway,above] {$1-p$};
            \draw[-{Latex[length=2mm]}] (tree-4-2) -- (tree-3-3) node[midway,above] {$p$};
            \draw[-{Latex[length=2mm]}] (tree-4-2) -- (tree-5-3) node[midway,above] {$1-p$};
            \node[below=0.5cm] at (tree-6-1) {$t$};
            \node[below=0.5cm] at (tree-6-2) {$t+\Delta t$};
            \node[below=0.5cm] at (tree-6-3) {$t+2\Delta t$};
            \node[below=1.1cm] at (tree-6-3) {$(=T)$};
            \draw[row sep=0.25cm] (tree-6-1.mid) -- (tree-6-3.mid) ;
            \draw[<->, line width=0.5mm] (tree-1-4.south west) -- (tree-3-4.north west) node[midway,right=1mm] {$2\sigma\sqrt{\Delta t}$};
            \draw[<->, line width=0.5mm] (tree-3-4.south west) -- (tree-5-4.north west) node[midway,right=1mm] {$2\sigma\sqrt{\Delta t}$};
        \end{tikzpicture}
        \vspace*{1mm}
        \caption{A CRR tree for $n = 2$.}
        \label{fig:btree}
    \end{figure}

\section{The Binomial-Trinomial Tree}\label{sec:preliminaries-btt}
        
    A binomial-trinomial tree (BTT) starts with a trinomial structure in the first 
    time step and continues on with a binomial one. Below, we illustrate how to 
    construct a BTT with a barrier $b$ in logarithm from $t$ to $T$ with reference to 
    Figure~\ref{fig:bino_tri_lattice_with_barrier}. First, lay down the lattice 
    underlying the \emph{binomial} structure, where one of the horizontal lines is 
    aligned with $b$. The horizontal lines of the lattice are equispaced at 
    $\sigma\sqrt{\Delta t}$ apart. Let $s_z$ denote the logarithmic stock price of 
    node $z$ and define $s_\text{X} - s_\text{Y}$ as the logarithmic return from node 
    X to node Y. Define $\theta \equiv (\mu-\sigma^2/2)\,\Delta t$ as the expected 
    logarithmic return for continuous-time model~(\ref{eq:log_stock_process}) at time 
    $\Delta t$. The trinomial structure for the first time step at the root node R is 
    built as follows. Name R's top, middle, and bottom successor nodes A, B, and C,
    respectively. Place node B on the grid point at time $\Delta t$ so that
    $\hat{\theta} \equiv s_\text{B}-s_\text{R}$ is closest to $\theta$, breaking ties 
    in favor of the lower node. Place node A (node C) $2\sigma\sqrt{\Delta t}$ higher
    (lower, resp.) than node B. 

    \begin{figure}[!t]
        \centering
        \resizebox{0.9\textwidth}{!}{
          \begin{tikzpicture}[myset/.list={(5,2),(7,2),(9,2),
              (4,3),(6,3),(8,3),(10,3),
              (3,4),(5,4),(7,4),(9,4),(11,4),
              (2,5),(4,5),(6,5),(8,5),(10,5),(12,5),
              (1,6),(3,6),(5,6),(7,6),(9,6),(13,6)}, fillset/.list={(11,6)},
              fontset/.list={(15,1),(15,2),(15,3),(15,4),(15,5),(15,6)},
              left extended/.style={shorten <=-#1},
              left extended/.default=1cm,
              right extended/.style={shorten >=-#1},
              right extended/.default=1cm]
          \matrix (tree) [%
            nodes in empty cells,
            matrix of nodes,
            nodes={anchor=center,minimum size=0.75cm},
            column sep=1.5cm,
            row sep=0.5cm,
            ampersand replacement=\&
          ]
          {
                \&   \& \& \& \&   \& \\
                \&   \& \& \& \&   \& \\
                \&   \& \& \& \&   \& \\
                \&   \& \& \& \&   \& \\
                \& A \& \& \& \&   \& \\
                \&   \& \& \& \&   \& \\
                \& B \& \& \& \&   \& \\
                \&   \& \& \& \&   \& \\
                \& C \& \& \& \&   \& \\
                \&   \& \& \& \&   \& \\
                \&   \& \& \& \&   \& \\
                \&   \& \& \& \&   \& \\
                \&   \& \& \& \&   \& \\
              $|$ \& $|$ \& $|$ \& $|$ \& $|$ \& $|$ \& \\
          };
          \begin{pgfonlayer}{background}
              \draw[line width=0.5mm, left extended] (tree-11-1) -- (tree-11-6);
              \foreach \y in {1,...,6}{
                  \draw[dashed, left extended, right extended] (tree-1-\y) -- (tree-13-\y);
              }
              \foreach \x in {1,2,3,4,6,8,10,11,12,13}{
                  \draw[dashed, left extended, right extended] (tree-\x-2) -- (tree-\x-6);
              }
              \foreach \x in {5,7,9}{
                  \draw[dashed, left extended=1.06cm, right extended] (tree-\x-2) -- (tree-\x-6);
              }
          \end{pgfonlayer}
          \node[right=1cm,font=\large] at (tree-11-6) {$b$};
          \node[right=1cm,font=\large] at (tree-1-6) {$b + 10\sigma\sqrt{\Delta t}$};
          \node[right=1cm,font=\large] at (tree-2-6) {$b + 9\sigma\sqrt{\Delta t}$};
          \node[right=1cm,font=\large] at (tree-3-6) {$b + 8\sigma\sqrt{\Delta t}$};
          \node[right=1cm,font=\large] at (tree-4-6) {$b + 7\sigma\sqrt{\Delta t}$};
          \node[right=1cm,font=\large] at (tree-5-6) {$b + 6\sigma\sqrt{\Delta t}$};
          \node[right=1cm,font=\large] at (tree-6-6) {$b + 5\sigma\sqrt{\Delta t}$};
          \node[right=1cm,font=\large] at (tree-7-6) {$b + 4\sigma\sqrt{\Delta t}$};
          \node[right=1cm,font=\large] at (tree-8-6) {$b + 3\sigma\sqrt{\Delta t}$};
          \node[right=1cm,font=\large] at (tree-9-6) {$b + 2\sigma\sqrt{\Delta t}$};
          \node[right=1cm,font=\large] at (tree-10-6) {$b + \sigma\sqrt{\Delta t}$};
          \node[right=1cm,font=\large] at (tree-13-6) {$b - 2\sigma\sqrt{\Delta t}$};
          \node[right=1cm,font=\large] at (tree-12-6) {$b - \sigma\sqrt{\Delta t}$};
          \node[below=0.5cm,font=\large] at (tree-14-1) {$t$};
          \node[below=0.5cm,font=\large] at (tree-14-2) {$t + \Delta t$};
          \node[below=0.5cm,font=\large] at (tree-14-3) {$t + 2\Delta t$};
          \node[below=0.5cm,font=\large] at (tree-14-4) {$t + 3\Delta t$};
          \node[below=0.5cm,font=\large] at (tree-14-5) {$t + 4\Delta t$};
          \node[below=0.5cm,font=\large] at (tree-14-6) {$t + 5\Delta t$};
          \node[below=1.2cm,font=\large] at (tree-14-6) {$(=T)$};
          \draw[-{Latex[length=2mm]}] (tree-8-1.north) -- (tree-5-2) node [midway,above=3mm,font=\large] {$P_u$}; 
          \draw[-{Latex[length=2mm]}] (tree-8-1.north) -- (tree-7-2) node [midway,below,font=\large] {$P_m$}; 
          \draw[-{Latex[length=2mm]}] (tree-8-1.north) -- (tree-9-2) node [midway,below,font=\large] {$P_d$};
          \node[draw,circle,minimum size=0.75cm,fill=white,font=\large,anchor=center] at (tree-8-1.north) {R};
          \draw[-{Latex[length=2mm]}] (tree-5-2) -- (tree-4-3); 
          \draw[-{Latex[length=2mm]}] (tree-5-2) -- (tree-6-3); 
          \draw[-{Latex[length=2mm]}] (tree-7-2) -- (tree-6-3); 
          \draw[-{Latex[length=2mm]}] (tree-7-2) -- (tree-8-3); 
          \draw[-{Latex[length=2mm]}] (tree-9-2) -- (tree-8-3); 
          \draw[-{Latex[length=2mm]}] (tree-9-2) -- (tree-10-3); 
          \draw[-{Latex[length=2mm]}] (tree-4-3) -- (tree-3-4); 
          \draw[-{Latex[length=2mm]}] (tree-4-3) -- (tree-5-4); 
          \draw[-{Latex[length=2mm]}] (tree-6-3) -- (tree-5-4); 
          \draw[-{Latex[length=2mm]}] (tree-6-3) -- (tree-7-4); 
          \draw[-{Latex[length=2mm]}] (tree-8-3) -- (tree-7-4); 
          \draw[-{Latex[length=2mm]}] (tree-8-3) -- (tree-9-4); 
          \draw[-{Latex[length=2mm]}] (tree-10-3) -- (tree-9-4); 
          \draw[-{Latex[length=2mm]}] (tree-10-3) -- (tree-11-4); 
          \draw[-{Latex[length=2mm]}] (tree-3-4) -- (tree-2-5); 
          \draw[-{Latex[length=2mm]}] (tree-3-4) -- (tree-4-5); 
          \draw[-{Latex[length=2mm]}] (tree-5-4) -- (tree-4-5); 
          \draw[-{Latex[length=2mm]}] (tree-5-4) -- (tree-6-5); 
          \draw[-{Latex[length=2mm]}] (tree-7-4) -- (tree-6-5); 
          \draw[-{Latex[length=2mm]}] (tree-7-4) -- (tree-8-5); 
          \draw[-{Latex[length=2mm]}] (tree-9-4) -- (tree-8-5); 
          \draw[-{Latex[length=2mm]}] (tree-9-4) -- (tree-10-5); 
          \draw[-{Latex[length=2mm]}] (tree-11-4) -- (tree-10-5); 
          \draw[-{Latex[length=2mm]}] (tree-11-4) -- (tree-12-5); 
          \draw[-{Latex[length=2mm]}] (tree-2-5) -- (tree-1-6); 
          \draw[-{Latex[length=2mm]}] (tree-2-5) -- (tree-3-6); 
          \draw[-{Latex[length=2mm]}] (tree-4-5) -- (tree-3-6); 
          \draw[-{Latex[length=2mm]}] (tree-4-5) -- (tree-5-6); 
          \draw[-{Latex[length=2mm]}] (tree-6-5) -- (tree-5-6); 
          \draw[-{Latex[length=2mm]}] (tree-6-5) -- (tree-7-6); 
          \draw[-{Latex[length=2mm]}] (tree-8-5) -- (tree-7-6); 
          \draw[-{Latex[length=2mm]}] (tree-8-5) -- (tree-9-6); 
          \draw[-{Latex[length=2mm]}] (tree-10-5) -- (tree-9-6); 
          \draw[-{Latex[length=2mm]}] (tree-10-5) -- (tree-11-6); 
          \draw[-{Latex[length=2mm]}] (tree-12-5) -- (tree-11-6); 
          \draw[-{Latex[length=2mm]}] (tree-12-5) -- (tree-13-6); 
          \draw (tree-14-1.mid) -- (tree-14-6.mid);
          %\draw[dashed, line width=0.5mm] (tree-7-6.east) -- (tree-7-7.east) ;
          %\draw[dashed, line width=0.5mm] (tree-9-6.east) -- (tree-9-7.east) ;
          %\draw[latex-latex, line width=0.5mm] (tree-7-7.west) -- (tree-9-7.west) node[midway,right=1mm,font=\large] {$2\sigma\sqrt{\Delta t}$};
        \end{tikzpicture}}
        \vspace*{1mm}
        \caption{A BTT with a barrier $b$. It starts with a trinomial structure. The part 
                 after time $t+\Delta t$ features a binomial structure.}
        \label{fig:bino_tri_lattice_with_barrier}
    \end{figure}

    Define
    \begin{align*}
        \alpha\quad &\equiv\quad \hat{\theta} - \theta + 2\sigma \sqrt{\Delta t}, \numberthis \label{eq:alpha} \\[3pt]
         \beta\quad &\equiv\quad \hat{\theta} - \theta, \numberthis \label{eq:beta} \\[3pt]
        \gamma\quad &\equiv\quad \hat{\theta} - \theta - 2\sigma \sqrt{\Delta t} \numberthis \label{eq:gamma}
    \end{align*}
    (see~Figure~\ref{fig:tri}). Because the horizontal lines are 
    $\sigma\sqrt{\Delta t}$ apart, we have
    \begin{equation}
        - \sigma\sqrt{\Delta t}\ \leq\ \beta\ \leq\ \sigma\sqrt{\Delta t}.
        \label{eq:beta_range}
    \end{equation}
    After the trinomial structure is established, grow a binomial structure from 
    nodes A, B, and C at time $\Delta t$ through time $T$ on the underlying grid (see 
    Figure~\ref{fig:bino_tri_lattice_with_barrier}). Note that the position of node B 
    can be written as $b + i^*\,\sigma\sqrt{\Delta t}$ where
    \begin{equation*}
        i^* \quad=\quad \operatorname*{argmin}_{i\in Z} \left\vert\,b + i\,\sigma\sqrt{\Delta t} - \theta\,\right\vert
    \end{equation*}
    and, in case the minimum value is reached at two $i$s, the lower $i$ is chosen 
    because the algorithm breaks ties in favor of the lower node.

    \begin{figure}[!t]
        \centering
        \begin{tikzpicture}
            [myset/.list={(6,1),(1,3),(4,3),(7,3)},
              left extended/.style={shorten <=-#1},
              left extended/.default=1cm,
              right extended/.style={shorten >=-#1},
              right extended/.default=1cm]
          \matrix (tree) [%
            nodes in empty cells,
            matrix of nodes,
            nodes={anchor=center,minimum size=0.75cm},
            column sep=2cm,
            row sep=0.3cm,
            ampersand replacement=\&
          ]
          {
                   \& \&  A  \&   \& \\
                   \& \&     \&   \& \\
                   \& \&     \&   \& \\
                   \& \&  B  \&   \& \\
                   \& \&     \&   \& \\
              R    \& \&     \&   \& \\
                   \& \&  C  \&   \& \\
              $|$  \& \& $|$ \&   \& \\
          };
            \node[below=0.5cm,font=\large] at (tree-8-1) {$t$};
            \node[below=0.5cm,font=\large] at (tree-8-3) {$t+\Delta t$};
            \draw[-{Latex[length=3mm]}] (tree-6-1) -- (tree-1-3) node [midway,above=3mm,font=\large] {$P_u$}; 
            \draw[-{Latex[length=3mm]}] (tree-6-1) -- (tree-4-3) node [midway,below,font=\large] {$P_m$}; 
            \draw[-{Latex[length=3mm]}] (tree-6-1) -- (tree-7-3) node [midway,below,font=\large] {$P_d$}; 
            \draw[line width=0.5mm, left extended=1.8cm] (tree-1-4) -- (tree-1-5) node [right=0.5mm,font=\large] {$\hat{\theta} + 2\sigma\sqrt{\Delta t}$}; 
            \draw[line width=0.5mm] (tree-4-4) -- (tree-4-5) node [right=0.5mm,font=\large] {$\hat{\theta}$}; 
            \draw[dashed, line width=0.5mm, left extended=1.8cm] (tree-5-4) -- (tree-5-5) node [right=0.5mm,font=\large] {$\theta$}; 
            \draw[line width=0.5mm, left extended=1.8cm] (tree-7-4) -- (tree-7-5) node [right=0.5mm,font=\large] {$\hat{\theta} - 2\sigma\sqrt{\Delta t}$}; 
            \draw[latex-latex, line width=0.5mm, left extended=0.2cm, right extended=0.2cm] (tree-4-4.south east) -- (tree-5-4.north east) node [midway,right=1mm,font=\large] {$|\ \beta\ |$}; 
            \draw[latex-latex, line width=0.5mm, left extended=0.2cm, right extended=0.2cm] (tree-1-4.south west) -- (tree-5-4.north west) node [midway,right=1mm,font=\large] {$|\ \alpha\ |$}; 
            \draw[latex-latex, line width=0.5mm, left extended=0.2cm, right extended=0.2cm] (tree-5-4.south west) -- (tree-7-4.north west) node [midway,right=1mm,font=\large] {$|\ \gamma\ |$}; 
            \draw[-] (tree-8-1.mid) -- (tree-8-3.mid); 
        \end{tikzpicture}
        \caption{The trinomial structure of a BTT.}
        \label{fig:tri}
    \end{figure}
    
    Obtain the branching probabilities by solving the following linear equations: 
    \begin{empheq}{align}
        P_u\alpha + P_m\beta + P_d\gamma\quad   &=\quad 0, \label{eq:mean} \\
        P_u\alpha^2 + P_m\beta^2 + P_d\gamma^2\quad &=\quad \text{Var} \label{eq:variance}, \\
        P_u + P_m + P_d\quad &=\quad 1 \label{eq:probs},
    \end{empheq}
    where Var $\equiv \sigma^2\Delta t$ is the variance of the logarithmic return 
    after $\Delta t$ time under the process~(\ref{eq:log_stock_process}). 
    Equations~(\ref{eq:mean}) and (\ref{eq:variance}) match the mean and variance of 
    the underlying model. Equation~(\ref{eq:probs}) makes sure they are 
    probabilities when they are also non-negative. By Cramer's rule, 
    $P_u = \Delta_u/\Delta$, $P_m = \Delta_m/\Delta$, and $P_d = \Delta_d/\Delta$, 
    where
    \begin{align*}
        \Delta  \quad &\equiv \quad \left(\beta - \alpha\right)\left(\gamma - \beta\right)\left(\gamma - \alpha\right) = -16\left(\sigma\sqrt{\Delta t}\right)^3, \numberthis \label{eq:det} \\
        \Delta_u\quad &\equiv \quad \left(\beta\gamma  + \text{Var}\right)\left(\gamma - \beta\right) = -2\sigma\sqrt{\Delta t}(\beta\gamma  + \text{Var}), \numberthis \label{eq:detu} \\
        \Delta_m\quad &\equiv \quad \left(\alpha\gamma + \text{Var}\right)\left(\alpha - \gamma\right) = 4\sigma\sqrt{\Delta t}(\alpha\gamma + \text{Var}), \numberthis \label{eq:detm} \\
        \Delta_d\quad &\equiv \quad \left(\alpha\beta  + \text{Var}\right)\left(\beta  - \alpha\right) = -2\sigma\sqrt{\Delta t}(\alpha\beta  + \text{Var}). \numberthis \label{eq:detd}
    \end{align*}
    Appendix~\ref{appendix:validity_of_p} proves the validity of the three 
    probabilities. Note that $\beta$ determines the actual positions of nodes A, B, 
    and C on the lattice, thus the whole binomial structure by extension. From 
    formulas~(\ref{eq:alpha})--(\ref{eq:gamma}) and 
    (\ref{eq:det})--(\ref{eq:detd}):
    \begin{align*}
        P_u \quad &=\quad \frac{\left(\beta-\sigma\sqrt{\Delta t}\right)^2}{8\sigma^2\Delta t}, \numberthis \label{eq:pu} \\
        P_m \quad &=\quad \frac{3\sigma^2\Delta t - \beta^2}{4\sigma^2\Delta t}, \numberthis \label{eq:pm} \\
        P_d \quad &=\quad \frac{\left(\beta+\sigma\sqrt{\Delta t}\right)^2}{8\sigma^2\Delta t} \numberthis \label{eq:pd}.
    \end{align*}
    Note that $P_u$, $P_m$, and $P_d$ are quadratic functions of $\beta$, which is a 
    critical property in solving the implied barrier later. 

    Incidentally, making the middle branch track the expected logarithmic return 
    instead of simply being flat, a standard choice, solves the notorious 
    barrier-too-close problem (see~\cite{BTT}). This problem refers to the need of 
    very large $n$ for trees when the barrier level of a barrier option is very close 
    to the initial stock price in order to maintain accuracy. The experiments will 
    confirm that our algorithms obtain implied barriers and option prices fast and 
    accurately even when the barrier is extremely close to the initial stock price.