% !TeX root = ../main.tex

\chapter{Introduction}\label{chap:introduction}

\section{Background and Motivations}\label{sec:introduction-background}

    Many real-world systems operate under uncertainty and often exhibit significant 
    changes when certain thresholds are crossed. These thresholds can mark the 
    transition from normalcy to failure, from equilibrium to instability, or from 
    solvency to insolvency. When such thresholds are exerted over a period of time, 
    either continuously or at discrete time points, they are called barriers. 
    Barriers play a critical role in understanding the dynamics of real-world systems 
    because they govern when and how significant changes occur abruptly 
    (see~\cite{Scheffer2009} and \cite{SchefferNature2009}). When barriers are 
    latent, accurately and efficiently obtaining them is essential to preventing 
    irreversible system transitions, supporting early-warning mechanisms, and 
    enhancing overall decision-making under uncertainty.

    This thesis focuses on barriers in credit analysis (see~\cite{Duffie2003} and 
    \cite{Lando2004}). Specifically, we examine the barrier that separates a 
    company's creditworthy status from the onset of a credit event. It is called the 
    default boundary. In practice, the default boundary itself is not directly 
    observable (see~\cite{Dionne2012}). Instead, financial indicators such as asset 
    value, leverage ratio, or working capital are commonly monitored for 
    deteriorating financial conditions (see~\cite{Altman1968}). When an indicator 
    falls to a critical level, the company may become unable to meet its 
    obligations, resulting in a credit event. For instance, a significant decrease in 
    asset value relative to debt can weaken creditors' confidence and raise the 
    borrowing cost. As another example, a substantial reduction in working capital, 
    such as declining cash reserves or rising short-term payables, may force the 
    company to seek emergency funding under unfavorable terms. These scenarios 
    illustrate how disappointing financial indicators can lead to deteriorating 
    credit quality, which may result in credit events such as debt restructuring or 
    even bankruptcy. In such cases, the default boundary is crossed. 

    A company's default boundary is heavily influenced by its liability structure, 
    which involves many kinds of obligations such as debt, taxes, and accounts 
    payable (see~\cite{Liu2016}). Straight bonds have clear payment schedules and 
    fixed maturity dates. If a company's cash is insufficient to meet the required 
    payments, default can occur. Debts with embedded options or redemption features 
    do not even have well-defined maturities, which introduces uncertainty about the 
    timing and size of future obligations (see~\cite{Ian2007}). Debt issues 
    furthermore interact with each other to complicate the default boundary 
    (see~\cite{Wang2014}). As for taxes, a change in corporate tax rates can 
    significantly alter the present value of the interest-tax shield and therefore 
    affect a company's leverage ratio (see~\cite{Graham2000}). Altogether, these 
    factors show how a company's liability structure influences its default boundary. 

    The default boundary can also be influenced by strategic considerations. 
    Theoretically, a company is considered solvent if its assets exceed its 
    liabilities because it can, in principle, liquidate its assets to repay its 
    liabilities. But a company's equity holders may choose to default strategically 
    even when the company's assets suffice to cover its liabilities 
    (see~\cite{AndersonSundaresanTychon1996}). This is because default may confer 
    certain advantages over the repayment of debts. For example, when a company is in 
    distress, the equity holders may threaten to file for bankruptcy in order to 
    renegotiate the terms of its debt. While debt holders are eligible to seize the 
    company's assets upon default, this process is costly and time-consuming, which 
    weakens the debt holders' bargaining position. As a result, the equity holders 
    may be able to negotiate a lower interest rate or even a partial cancellation. 
    %These complexities make it nontrivial to determine a default boundary. 

    Given that the default boundary is not directly observable and is influenced by 
    multiple interacting factors, inferring this boundary from data becomes an 
    attractive alternative (see~\cite{Avellaneda2001} and \cite{Hull2001}). The 
    resulting default boundary is called the \emph{implied} barrier by 
    \cite{Brockman2003}, which is the main topic of this thesis. When the implied 
    barrier is obtained from publicly available data, it reflects market consensus. 
    This thesis will obtain the implied barrier from publicly available default 
    probabilities, which refer to the probabilities that a company will default over 
    periods of time. Specifically, the implied barrier is the default boundary that 
    reproduces the default probabilities under a structural model. 

    Structural models are a class of credit risk models based on the option-pricing 
    framework; as a result, corporate securities are contingent claims on a company's 
    assets (see~\cite{Lando2004}). While initially developed to price corporate 
    liabilities, structural models offer an economic framework for credit analysis. 
    These models make three assumptions: (1) the company's asset value follows a 
    stochastic process, (2) default occurs if the asset value drops to a default 
    boundary, and, if implicitly, (3) the default boundary is a function of asset 
    value and time. 

    The first structural model is proposed by \cite{Merton1974}. He assumes a 
    company's asset value follows a geometric Brownian motion and that it has only 
    zero-coupon bonds (ZCBs) with the same maturity as the liabilities. In Merton's 
    model, default occurs when the asset value falls to the total face value of the 
    ZCBs at their maturity. The framework can be extended to any debt 
    structure in principle.
    
    In reality, companies can default anytime, contrary to Merton's assumption. 
    \cite{FPM} address this issue by proposing a first-passage-time structural model. 
    They specify default as the \emph{first} time the asset value touches the 
    default boundary. Like Merton, they assume the company holds same-maturity 
    ZCBs but now with safety covenants. The covenants provide bondholders with the 
    right to initiate bankruptcy proceedings or enforce a reorganization if they are 
    violated. Therefore, the safety covenants give rise to the default boundary.

    The determination and, when applicable, the functional form of the default 
    boundary play a critical role in structural models. Before delving into our 
    approach, it is useful to examine how default boundaries are typically handled 
    within structural models. One common approach is to consider the book value of 
    the outstanding liabilities and assume a parametric form. The simplest form is a 
    constant default boundary (see~\cite{Kim1993}, \cite{Leland1996}, 
    \cite{Brockman2003}, and \cite{KMV}). However, a constant default boundary can be 
    internally inconsistent. Consider $p(t_1)$ and $p(t_2)$ as the cumulative 
    probabilities of a company defaulting at or before times $t_1$ and $t_2$, 
    respectively, where $t_1 < t_2$. They are likely to result in two different 
    levels of default boundary $b_1$ and $b_2$, respectively, where $b_1$ covers the 
    time from now through $t_1$ whereas $b_2$ covers the time from now through the 
    later $t_2$. So the period from now to time $t_1$ is associated with two 
    different levels. Other than the constant default boundary, the exponential 
    default boundary is also popular (see~\cite{FPM}, \cite{Longstaff1995}, and 
    \cite{Zhou2000}). Instead of assuming a parametric form, 
    \cite{AndersonSundaresan1996} apply game theory to endogenously determine the 
    default boundary. Their default boundary is the equilibrium of a game in which 
    equity holders and debt holders interact strategically. 
    
    As mentioned above, this thesis infers the default boundary from publicly 
    available default probabilities under a structural model; this boundary is the 
    implied barrier. We define the default probability over the time horizon 
    $(\,0, T\,]$ as the probability of the company's asset value, starting above the 
    boundary, hitting the barrier at or before $T$. The survival probability, which 
    equals one minus the default probability, equals the expected payoff of a special 
    barrier option: The down-and-out digital barrier call when the strike price 
    equals the barrier (DODC). A DODC pays one dollar at maturity $T$ if and only if 
    the underlying asset's price never touches the barrier by $T$. Throughout this 
    thesis, the underlying asset is assumed to be a stock, except in 
    Chapter~\ref{chap:application}.
    %Thus, the default probability over the time horizon $(\,0, T\,]$ equals one minus 
    %the expected payoff of a DODC at maturity $T$.

    As there are no closed-form formulas for the implied barrier, any numerical 
    methods to find it must repeatedly calculate the expected payoff of a DODC and 
    compare it against the target survival probability until the difference is 
    acceptably small. When the barrier is a constant, the expected payoff can be 
    easily calculated through a closed-form formula 
    (see~\cite{ReinerRubinstein1991}). However, when the barrier has a more complex 
    form, such as a step function, the closed-form formula is hard to compute in 
    general as it involves a multiple integral (see~\cite{Lee2021}). We adopt tree 
    algorithms to compute the expected payoff of the DODC with a step barrier because 
    they are efficient. This thesis focuses on the step-barrier case because a step 
    barrier can approximate any continuous barrier.

    With the implied barrier in place, we can apply it to corporate default 
    prediction. The task is to identify whether a company will experience a credit 
    event within a time horizon. The importance of having an effective default 
    prediction scheme is self-evident: The cost of default can be high. For example, 
    in 2007, ``irrational exuberance'' in the market led to an underestimation of 
    default risk and resulted in the overpricing of such derivatives as 
    collateralized debt obligations (CDOs) (see~\cite{Shiller2009}). This became a 
    crucial factor in the ensuing global financial crisis and cost more than a 
    trillion dollars in national output (see~\cite{Atkinson2013} and 
    \cite{Nickerson2017}).

    Typical default prediction schemes provide default probabilities 
    (see~\cite{Ohlson1980}, \cite{Zmijewski1984}, \cite{Shumway2001}, 
    \cite{Chava2004}, and \cite{Campbell2008}). While they can be used to rank 
    companies in terms of default risk, sometimes a binary classification is needed 
    for decision-making. For example, regulators may need to classify banks as either 
    at risk of failure or not before taking supervisory actions 
    (see~\cite{Martin1977}). Of course, a default probability can be converted into a 
    binary prediction, i.e., default or no-default, by applying a cutoff threshold. 
    The threshold may be chosen in an ad hoc manner (often 0.5) or based on some 
    principles (see~\cite{Ohlson1980}, \cite{Zmijewski1984}, \cite{Palepu1986}, and 
    \cite{Barboza2017}).
    %In fact, there is evidence suggesting that 0.5 is not an appropriate threshold 
    %for default probabilities. This is because default is a rare event and a 
    %threshold of 0.5 assumes a symmetric loss function for default group and 
    %non-default group, which does not reflect the reality (see~\cite{Ohlson1980}).

    Unlike thresholds, the implied barrier provides an alternative way to turn 
    default probabilities into binary default predictions. It is furthermore 
    economically reasonable. The idea is to compare the company's future expected 
    asset value with the implied barrier. Default is predicted if and only if the 
    expected asset value ever touches the implied barrier. Since the implied barrier 
    is inferred from default probabilities, which in turn rely on market and 
    accounting data, our scheme focuses on publicly listed companies, for which such 
    information is available. Finally, this thesis proposes a new scheme to evaluate
    the performance of binary default prediction schemes.
    
\section{Literature Review}\label{sec:introduction-literature_review}

\subsection{Algorithms for Implied Barrier}\label{sec:introduction-literature_review-algorithms}
   
    In structural models, the default probability is equivalent to the probability of 
    the company's asset value hitting the barrier from above, i.e., the 
    barrier-hitting probability. Moreover, the barrier-hitting probability equals one 
    minus the DODC's expected payoff. An option's expected payoff is closely related 
    to its price: The expected payoff is computed under the real-world measure, while 
    the option price is evaluated under the risk-neutral measure and involves 
    discounting. Thus, if one can compute one, one can compute the other with the 
    same computational time.

    A common approach to inferring the implied barrier is to numerically invert 
    closed-form formulas for either the barrier-hitting probability or the barrier 
    option price. Under the \citetalias{BS} model, formulas for the barrier-hitting 
    probability are available for the constant-barrier case 
    (see~\cite{Ingersoll1987}) and the step-barrier case (see~\cite{Lee2021}). 
    Formulas for the barrier option price are available for the cases of constant 
    barrier (see~\cite{Merton1974} and \cite{Reiner1991}), discrete constant barrier 
    (see~\cite{Broadie1997} and \cite{Kou2003}), partial and window constant barriers 
    (see~\cite{Heynen1994}, \cite{Hui1997}, and \cite{Armstrong2001}), piecewise 
    exponential barrier (see~\cite{Kunitomo1992}), step barrier (see~\cite{Lee2021}), 
    continuously differentiable barrier (see~\cite{Rogers1997}), and a mixture of 
    continuously and discretely monitored barrier (see~\cite{Chen2010}). Note that 
    any barrier that is not constant can be viewed as a time-varying barrier, which 
    is also called a moving barrier in this thesis. Figure~\ref{fig:barriers} shows 
    five common types of barriers.

    \begin{figure}[!t]
        \centering
        \begin{subfigure}[b]{0.48\textwidth}
            \centering
            \begin{tikzpicture}
                \begin{axis}[
                        width=\textwidth,
                        height=0.75\textwidth,
                        xtick = {0.5,3.5},
                        xticklabels = {$0$,$T$},
                        xticklabel style={text height=1.5ex, font=\normalsize},
                        ytick = {1.5},
                        yticklabels = {$b$},
                        yticklabel style={font=\normalsize},
                        xmin=-0.5,xmax=4,
                        ymin=0,ymax=3,
                    ]
                    \addplot[line,-,domain=0.5:3.5]{1.5}; %node [midway,above] {$b_1$};
                    \addplot[solid] coordinates{(0.5,1.5)(3.5,1.5)};
                \end{axis}
            \end{tikzpicture}
            \caption{Constant barrier.}
        \end{subfigure}
        \hfill
        \begin{subfigure}[b]{0.48\textwidth}
            \centering
            \begin{tikzpicture}
                \begin{axis}[
                        width=\textwidth,
                        height=0.75\textwidth,
                        xtick = {0.5,1.5,2.5,3.5},
                        xticklabels = {$0$,$t_1$,$t_2$,$T$},
                        xticklabel style={text height=1.5ex, font=\normalsize},
                        ytick = {2.4,0.8,2},
                        yticklabels = {$b_1$,$b_2$,$b_3$},
                        yticklabel style={font=\normalsize},
                        xmin=-0.5,xmax=4,
                        ymin=0,ymax=3,
                    ]
                    \addplot[line,-,domain=0.5:1.5]{2.4}; %node [midway,above] {$b_1$};
                    \addplot[line,-,domain=1.5:2.5]{0.8}; %node [midway,above] {$b_2$};
                    \addplot[line,-,domain=2.5:3.5]{2};   %node [midway,above] {$b_m$};
                    \addplot[hollow]coordinates{(0.5,2.4)(1.5,0.8)(2.5,2)};
                    \addplot[solid] coordinates{(1.5,2.4)(2.5,0.8)(3.5,2)};
                \end{axis}
            \end{tikzpicture}
            \caption{3-step barrier.}
        \end{subfigure}
        \hfill
        \begin{subfigure}[b]{0.48\textwidth}
            \centering
            \begin{tikzpicture}
                \begin{axis}[
                        width=\textwidth,
                        height=0.75\textwidth,
                        xtick = {0.5,1.5,2.5,3.5,4.5},
                        xticklabels = {$0$,$t_1$,$t_2$,$\cdots$,$T$},
                        xticklabel style={text height=1.5ex, font=\normalsize},
                        ytick = {1.5},
                        yticklabels = {$b$},
                        yticklabel style={font=\normalsize},
                        xmin=-0.5,xmax=5,
                        ymin=0,ymax=3,
                    ]
                    \addplot[solid] coordinates{(0.5,1.5)(1.5,1.5)(2.5,1.5)(4.5,1.5)};
                \end{axis}
            \end{tikzpicture}
            \caption{Discrete constant barrier.}
        \end{subfigure}
        \hfill
        \begin{subfigure}[b]{0.48\textwidth}
            \centering
            \begin{tikzpicture}
                \begin{axis}[
                        width=\textwidth,
                        height=0.75\textwidth,
                        xtick = {0.5,2,3.5},
                        xticklabels = {$0$,$t_1$,$T$},
                        xticklabel style={text height=1.5ex, font=\normalsize},
                        ytick = {1.5},
                        yticklabels = {$b$},
                        yticklabel style={font=\normalsize},
                        xmin=-0.5,xmax=4,
                        ymin=0,ymax=3,
                    ]
                    \addplot[line,-,domain=2:3.5]{1.5}; %node [midway,above] {$b_1$};
                    \addplot[solid] coordinates{(2,1.5)(3.5,1.5)};
                \end{axis}
            \end{tikzpicture}
            \caption{Partial constant barrier.}
        \end{subfigure}
        \hfill
        \begin{subfigure}[b]{0.48\textwidth}
            \centering
            \begin{tikzpicture}
                \begin{axis}[
                        width=\textwidth,
                        height=0.75\textwidth,
                        xtick = {0.5,1.5,2.5,3.5,4.5},
                        xticklabels = {$0$,$t_1$,$t_2$,$\cdots$,$T$},
                        xticklabel style={text height=1.5ex, font=\normalsize},
                        ytick = {1.5},
                        yticklabels = {$b$},
                        %yticklabels = {$ $,$ $,$ $},
                        yticklabel style={font=\normalsize},
                        xmin=-0.5,xmax=5,
                        ymin=0,ymax=3,
                    ]
                    \addplot[line,-,domain=1.5:2.5]{1.5}; %node [midway,above] {$b_1$};
                    \addplot[solid] coordinates{(1.5,1.5)(2.5,1.5)};
                \end{axis}
            \end{tikzpicture}
            \caption{Window constant barrier.}
        \end{subfigure}
        \caption{Popular barrier types. $t_i$s are the monitoring points and $T$ is 
                 the maturity date.}
        \label{fig:barriers}
    \end{figure}
    
    Recall that we focus on the step-barrier case because a step barrier can 
    approximate any continuous barrier. Formulas for step-barrier options involve 
    multiple integrals (\cite{Lee2021}). Extracting the implied barrier from them is 
    infeasible because the computational cost grows exponentially in the dimension. 
    Monte Carlo simulation is also not recommended for obtaining the implied barrier 
    because of its (1) stochastic and moderately large error, and (2) relative 
    inefficiency. A general and efficient alternative is thus desirable for finding 
    the implied step barrier. This thesis focuses on trees as a solution.
    
    Trees are commonly used to price financial derivatives when closed-form formulas
    are unavailable or too costly to evaluate (see~\cite{Lyuu2002}). A tree 
    discretizes the continuous process of the stock price. A properly calibrated tree 
    converges to the continuous-time model as $n$ increases, where $n$ is the tree's 
    number of time steps (see~\cite{Duffie1996}). \citetalias{CRR} propose the 
    first tree based on the BS model, known as the CRR tree. It can be used to price 
    various kinds of options. For example, the prices of a vanilla call by the CRR 
    tree are shown in Figure~\ref{fig:crr_call_price}. They clearly converge to the 
    true value given by the BS formula.

    The CRR tree has its drawbacks. When pricing barrier options in particular, the 
    CRR tree produces large oscillations, meaning slow convergence. This is caused 
    mostly by the misalignment between the barrier and the tree nodes, resulting in 
    the nonlinearity error (see~\cite{Figlewski1999}). 
    Figure~\ref{fig:crr_dodc_price} illustrates this problem when pricing a DODC. 
    
    \begin{figure}[!t]
        \centering
        \includegraphics[width = 0.7\textwidth]{figures/crr_call_price.pdf}
        \caption{The vanilla call prices by the CRR tree for $n = 10,20,30,\ldots,500$. 
                 %The initial stock price is 95, the strike 
                 %price is 100, the growth rate is $10\%$, the volatility is $25\%$, 
                 %and time to maturity is $1$ year. 
                 The dashed line 11.6573 gives the true value by the BS formula.}
        \label{fig:crr_call_price}
    \end{figure}
    
    \begin{figure}[!t]
        \centering
        \includegraphics[width = 0.7\textwidth]{figures/crr_dodc_price.pdf}
        \caption{The DODC prices by the CRR tree for $n = 10,20,30,\ldots,500$. 
                 %The initial stock price is 95, the strike price is 100, the 
                 %barrier is 90, the growth rate is $10\%$, the volatility is $25\%$, 
                 %and time to maturity is $1$ year. 
                 The dashed line 0.2023 gives the true value by \citetalias{ReinerRubinstein1991} formula.}
        \label{fig:crr_dodc_price}
    \end{figure}

    \cite{Boyle1994} and \cite{Ritchken1995} reduce the nonlinearity error by placing 
    critical tree nodes close to or on the barrier. \cite{BTT} propose the 
    binomial-trinomial tree (BTT), which starts with a trinomial structure and 
    continues on with a binomial structure. The trinomial part reduces the 
    nonlinearity error by aligning one of the tree levels in the binomial structure 
    with the barrier. The subsequent binomial part enables the use of fast 
    combinatorial formulas (see~\cite{Lyuu1998}) and the fast Fourier transform (FFT) 
    (see~Chapter~\ref{chap:algorithms}).

\subsection{Convergence Rates for Trees}\label{sec:introduction-literature_review-convergence}

    With trees, the implied barrier is determined by iteratively adjusting the 
    barrier level to match a target value such as the barrier-hitting probability or 
    the option's expected payoff. The efficiency of this process depends on how 
    quickly the algorithm computes the relevant quantity and compares it with the 
    target value, and on how rapidly the iterative procedure converges.
    
    \cite{CRR} not only propose the CRR tree to price options but also demonstrate 
    its convergence to the BS model. However, they do not address the convergence 
    rate. \cite{Walsh2003} and \cite{Chang2007} prove a linear convergence rate for 
    the vanilla European option price by the CRR tree. \cite{Leduc2020} prove a 
    quadratic convergence rate for the vanilla option price by the CRR tree after 
    Richardson's extrapolation is applied. \cite{Leduc2023} derive the convergence 
    rates for European path-independent options by binomial, trinomial, and 
    $m$-nomial trees. \cite{Lin2013} prove a linear convergence rate for the European 
    barrier call price given by the CRR tree when (1) the barrier is aligned with one 
    of the tree levels and (2) the strike price is equal to the barrier. The CRR tree 
    is unable to align with the barrier in general, in which case the nonlinearity 
    error makes the convergence erratic and slower than linear, as shown in 
    Figure~\ref{fig:crr_dodc_price}.
    
\subsection{Application to Default Prediction}\label{sec:introduction-literature_review-default_prediction}

    \cite{PeterDrucker} famously opines that ``forecasting is not a respectable human
    activity and not worthwhile beyond the shortest of periods.'' This thesis will 
    not contest the statement in the case of predicting company default; an almost
    perfect predictor for that task is neither our goal nor feasible. Instead, we 
    believe publicly available financial data, by constraining how the future 
    unfolds, already afford us very good guidance in predicting company default 
    within, say, the next 5 years.

    Default models can be used to price corporate debts or monitor the health of 
    financial entities such as banks and pension funds (see~\cite{Shumway2001}). A 
    model that can predict the default event of a financial entity with reasonable 
    accuracy narrows the information asymmetry between investors and management. A 
    good model may even spot profitable opportunities.
    
    Empirical analysis of default becomes active in the 1960s. \cite{Beaver1966} and
    \cite{Altman1968} apply discriminant analysis to accounting and economic ratios
    to gauge whether a company is financially sound. \cite{Ohlson1980} and
    \cite{Zmijewski1984} frame the problem with a logit model. For convenience, we
    refer to the above models as the traditional models. The advantage of traditional
    models is the transparent relation between the explanatory variables and the
    likelihood of default (see~\cite{Altman2019}). However, accounting data, being
    mostly backward looking, will obviously be strained in predicting future defaults
    (see~\cite{Hillegeist2004}). \cite{Shumway2001} refers to traditional models as 
    static models because they are time independent. These models rely on financial 
    indicators from a single period only, and therefore ignore companies' performance 
    over time. As a result, they might produce biased predictor variables and thus
    overlook default companies that had unfavorable indicators in periods other than 
    the one used in estimation. Outside those traditional approaches, default models 
    can be grouped into two categories: reduced-form models and structural models.

    Reduced-form models characterize the default event by an \emph{exogenous} point 
    process in which the default time is the first jump time of the process 
    (see~\cite{Duffie2003}). Initially, they focus on pricing corporate bonds by 
    specifying, explicitly or tacitly, a default intensity process and a recovery 
    rate process (see \cite{Litterman1991}, \cite{Jarrow1995}, \cite{Jarrow1997}, and 
    \cite{Duffie1999}). The default intensity gives rise to the default 
    probabilities. The recovery rate refers to the residual proportion of a financial 
    entity because investors are entitled to the remaining value after bankruptcy. 
    The default probabilities from reduced-form models have later been applied to 
    default prediction. \cite{Shumway2001}, \cite{Chava2004}, \cite{Campbell2008}, 
    and \cite{Bonfim2009} use hazard models to obtain default probabilities. 
    \cite{DSW} specify a doubly-stochastic formulation of the point process for 
    defaults and other types of firm exits caused by, for example, leveraged buyout, 
    seizure by regulator, delisting, or merger and acquisition. \cite{FIM} propose a 
    forward intensity approach to tackle the high-dimensional problem in \cite{DSW} 
    and make the implementation more efficient.
    
    Since the default event is triggered externally, reduced-form models lack an
    intuitive explanation of default. In contrast, in structural models, default 
    occurs when the company's asset value hits the default boundary from above 
    (recall Section~\ref{sec:introduction-background}). This means the default event 
    is determined \emph{endogenously} by the company's financial status. This thesis 
    adopts the structural models because they offer an intuitive explanation of 
    default events and provide a framework to explicitly talk about the default 
    boundary. 
    
\section{Contributions}\label{sec:introduction-contribution}
    
    The contributions of this thesis are as follows. First, we propose a highly
    efficient $O(n\,\text{log}\,n)$-time tree algorithm based on the BS model to 
    obtain the implied barrier, where $n$ denotes the tree's number of time steps. We 
    call it the core algorithm. When compared with the standard implementation, the 
    core algorithm is at least an order faster. The core algorithm finds the 
    \emph{exact} implied barrier for a tree from the barrier-hitting probabilities 
    when the barrier is a step function. The implied barrier is exact in the sense 
    that it reproduces the barrier-hitting probabilities for \emph{any} $n \geq 1$. 
    For simplicity, we omit the term ``exact'' in the thesis unless emphasis is 
    required. In the simplest constant-barrier case, the implied barrier is close to 
    the true barrier even with small $n$'s. Indeed, using $n = 1$ and $n = 3$ for
    extrapolation typically suffices to closely approximate the true barrier. 

    Second, we prove that the implied barrier by our core algorithm converges to its 
    true value at a rate of $O(n^{-1})$ when the barrier is a constant. The numerical 
    results suggest the same convergence rate holds when the barrier is a step 
    function. 
    
    Finally, we propose a binary default prediction scheme under structural models as 
    an application of the implied barrier. The core algorithm's accuracy and super 
    efficiency are the keys to handling this task because the dataset is huge. 
    Moreover, we propose a new scheme to evaluate binary default prediction schemes.
    The results suggest that the implied barrier performs well in predicting 
    corporate defaults for companies listed on U.S. exchanges. 

    The remainder of this thesis is structured as follows. 
    Chapter~\ref{chap:preliminaries} introduces the preliminaries and notations used 
    throughout the thesis. Chapter~\ref{chap:algorithms} presents the proposed tree
    algorithm for finding the implied barrier and its adaptations for option pricing.
    Chapter~\ref{chap:convergence} provides the mathematical proof of the linear 
    convergence rate of the implied barrier. Chapter~\ref{chap:application} explores 
    the practical applications of the implied barrier in corporate default 
    prediction. 